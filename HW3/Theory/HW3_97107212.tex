\documentclass[12pt]{article}
\usepackage{graphicx,import}
\usepackage[svgnames]{xcolor} 
\usepackage{fancyhdr}
\usepackage{subfig}
\usepackage{hyperref}
\usepackage{enumitem}
\usepackage{cite}
\usepackage[many]{tcolorbox}
\usepackage{listings }
\usepackage[a4paper, total={6in, 8in} , bottom = 25mm , top = 25mm, headheight = 1.25cm , includehead,includefoot,heightrounded ]{geometry}
\usepackage{afterpage}
\usepackage{amssymb}
\usepackage{pdflscape}
\usepackage{gensymb}
\usepackage{textcomp}
\usepackage{tikz,pgfplots}
\usepackage{xecolor}
\usepackage{rotating}
\usepackage{pdfpages}
\usepackage[Kashida]{xepersian}
\usepackage[T1]{fontenc}
\usepackage{tikz}
\usepackage[utf8]{inputenc}
\usepackage{PTSerif} 
\usepackage{seqsplit}

\usepackage[edges]{forest}

\usepackage{listings}
\usepackage{xcolor}

\hypersetup{
	colorlinks   = true, %Colours links instead of ugly boxes
	urlcolor     = blue, %Colour for external hyperlinks
	linkcolor    = blue, %Colour of internal links
	citecolor   = red %Colour of citations
}
 
\definecolor{codegreen}{rgb}{0,0.6,0}
\definecolor{codegray}{rgb}{0.5,0.5,0.5}
\definecolor{codepurple}{rgb}{0.58,0,0.82}
\definecolor{backcolour}{rgb}{0.95,0.95,0.92}
 
\NewDocumentCommand{\codeword}{v}{
\texttt{\textcolor{blue}{#1}}
}
\lstset{language=java,keywordstyle={\bfseries \color{blue}}}

\lstdefinestyle{mystyle}{
    backgroundcolor=\color{backcolour},   
    commentstyle=\color{codegreen},
    keywordstyle=\color{magenta},
    numberstyle=\tiny\color{codegray},
    stringstyle=\color{codepurple},
    basicstyle=\ttfamily\normalsize,
    breakatwhitespace=false,         
    breaklines=true,                 
    captionpos=b,                    
    keepspaces=true,                 
    numbers=left,                    
    numbersep=5pt,                  
    showspaces=false,                
    showstringspaces=false,
    showtabs=false,                  
    tabsize=2
}

\lstset{style=mystyle}

\settextfont[Scale=1.2 ,BoldFont={Bahij Nazanin-Bold.ttf} , ItalicFont = {IRNazaninIranic.ttf}]{Bahij Nazanin-Regular.ttf}
\setlatintextfont[Scale = 1.0]{Garamond}
\DefaultMathsDigits 
\DeclareMathSizes{11}{19}{13}{9} 
%\DeclareMathSizes{12}{14.4}{8}{9}





\newenvironment{changemargin}[2]{%
\begin{list}{}{%
\setlength{\topsep}{0pt}%
\setlength{\leftmargin}{#1}%
\setlength{\rightmargin}{#2}%
\setlength{\listparindent}{\parindent}%
\setlength{\itemindent}{\parindent}%
\setlength{\parsep}{\parskip}%
}%
\item[]}{\end{list}}


\definecolor{foldercolor}{RGB}{124,166,198}

\tikzset{pics/folder/.style={code={%
    \node[inner sep=0pt, minimum size=#1](-foldericon){};
    \node[folder style, inner sep=0pt, minimum width=0.3*#1, minimum height=0.6*#1, above right, xshift=0.05*#1] at (-foldericon.west){};
    \node[folder style, inner sep=0pt, minimum size=#1] at (-foldericon.center){};}
    },
    pics/folder/.default={20pt},
    folder style/.style={draw=foldercolor!80!black,top color=foldercolor!40,bottom color=foldercolor}
}

\forestset{is file/.style={edge path'/.expanded={%
        ([xshift=\forestregister{folder indent}]!u.parent anchor) |- (.child anchor)},
        inner sep=1pt},
    this folder size/.style={edge path'/.expanded={%
        ([xshift=\forestregister{folder indent}]!u.parent anchor) |- (.child anchor) pic[solid]{folder=#1}}, inner xsep=0.6*#1},
    folder tree indent/.style={before computing xy={l=#1}},
    folder icons/.style={folder, this folder size=#1, folder tree indent=3*#1},
    folder icons/.default={12pt},
}

\begin{document}


%%% title pages
\begin{titlepage}
\begin{center}
        
\vspace*{0.7cm}

\includegraphics[width=0.4\textwidth]{sharif1.png}\\
\vspace{0.5cm}
\textbf{ \Huge{\emph ‌سیگنال‌ها و سیستم‌ها} }\\
\vspace{0.5cm}
\textbf{ \Large{ تمرین سوم} }
\vspace{0.2cm}
       
 
      \large \textbf{دانشکده مهندسی کامپیوتر}\\\vspace{0.2cm}
    \large   دانشگاه صنعتی شریف\\\vspace{0.2cm}
       \large   ﻧﯿﻢ سال دوم 00-99 \\\vspace{0.2cm}
      \noindent\rule[1ex]{\linewidth}{1pt}
استاد:\\
    \textbf{{جناب آقای دکتر منظوری شلمانی}}


    \vspace{0.15cm}
نام و نام خانوادگی:\\

       
    \textbf{{امیرمهدی نامجو - 97107212}}
\end{center}
\end{titlepage}
%%% title pages


%%% header of pages
\newpage
\pagestyle{fancy}
\fancyhf{}
\fancyfoot{}
\cfoot{\thepage}
\chead{تمرین سوم}
\rhead{\includegraphics[width=0.1\textwidth]{sharif.png}}
\lhead{امیرمهدی نامجو}
%%% header of pages

\KashidaOff

\section{سوال اول}



$$x[n] =  u[n-1] + u[n-2] + ... u[n-N]$$

$$H(z) = \frac{z^{-1}}{1 - z^{-1}} + \frac{z^{-2}}{1 - z^{-1}} + \frac{z^{-3}}{1 - z^{-1}} + \cdots + \frac{z^{-N}}{1 - z^{-1}}$$

$$= \frac{z^{-N}}{1 -z^{-1}}(1+z+z^2 + \cdots + z^{N-1}) = \frac{z^{-N}}{1 -z^{-1}} (\frac{1-z^N}{1-z})$$

$$=\frac{z-z^{1-N}}{(z-1)^2}$$

\section{سوال دوم}

$$
X(z)=(1+2 z)\left(1+3 z^{-1}\right)\left(1-z^{-1}\right)
$$

$$X(z) = -3 z^{-2} - 4 z^{-1} + 5 + 2z$$

در نتیجه اگر تبدیل معکوس بگیریم داریم:
$$x(-1) = 2 ~~,~~ x(0) = 5 ~~,~~ x(1) = -4 , ~~x(2) = -3$$

و برای سایر نقاط هم $x(n) = 0$ است.

این تبدیل معکوس براساس ضرایب موجود در $X(z)$ بدست آمده است.



\section{سوال سوم}

$$
H(z)=\frac{1-z^{-1}}{1+\frac{3}{4} z^{-1}} ; ROC:‍ |z| > \frac{3}{4}
$$

دلیل ROC ذکر شده به خاطر LTI و علی بودن است.

\begin{enumerate}[label = \Alph*)]
	\item 
	$$H(z) = \frac{-4}{3} + \frac{\frac{7}{3}}{1 + \frac{3}{4} z^{-1}}$$
	
	$$\mathcal{Z}^{-1}(H(z))(n) = h[n] = \frac{-4}{3} \delta[n] + \frac{7}{3} (\frac{-3}{4})^n u[n]$$ 
	
	\item
	
	$$x[n] = (1/3)^n u[n] + u[-n -1]$$
	
	$$X(z) = \frac{1}{1 - (1/3) z^{-1}} - \frac{1}{1 - z^{-1}}; ROC: 1/3<|z|<1$$
	
	$$Y(z) = X(z) H(z) = (\frac{1}{1 - (1/3) z^{-1}} - \frac{1}{1 - z^{-1}}) (\frac{-4}{3} + \frac{\frac{7}{3}}{1 + \frac{3}{4} z^{-1}})$$$$ = \frac{\frac{-8}{13}}{1 - 1/3 z^{-1}} + \frac{\frac{8}{13}}{1 + 3/4 z^{-1}} ; ROC:|z| > \frac{3}{4}$$
	

	
	$$y[n] = \frac{-8}{13} (1/3)^n u[n] + \frac{8}{13} (-3/4)^{n} u[n] $$
	
	
	اشتراک ROC های $X$ و $H$ برابر $|z|>\frac{3}{4}$ است. و این ناحیه باید زیر مجموعه ROC نهایی ما باشد. با توجه به قطب‌های عبارت نهایی، ROC آن به صورت $|z| > \frac{3}{4}$ خواهد بود.
	
	
	
	\item
	سیستم LTI علی پایدار است اگر و تنها اگر قطب هایش از نظر اندازه کمتر مساوی $1$ باشند.
	
	در این مورد قطب $|z| = 3/4$ است که کمتر از یک است. پس پایدار است.
	
	
	
\end{enumerate}



\section{سوال چهارم}
$$
\begin{array}{l}
	x[n]=\left(\frac{1}{3}\right)^{n} u[n]+2^{n} u[-n-1] \\
	y[n]=5\left(\frac{1}{3}\right)^{n} u[n]-5\left(\frac{2}{3}\right)^{n} u[n]
\end{array}
$$



\begin{enumerate}[label = \Alph*)]
	\item 

$$X(z) = \frac{1}{1-1/3 z^{-1}} - \frac{1}{1- 2 z^{-1}} ; ROC: 1/3<|z|<2$$
$$Y(z)= \frac{5}{1 - 1/3 z^{-1}} - \frac{5}{1 - 2/3 z^{-1}} ; ROC:|z| >2/3$$

$$H(z) = \frac{Y(z)}{X(z)} =\frac{\frac{5}{1-\frac{1}{3 z}}-\frac{5}{1-\frac{2}{3 z}}}{\frac{1}{1-\frac{1}{3 z}}-\frac{1}{1-\frac{2}{z}}} =\frac{3 (z-2)}{3 z-2}$$
$$=3 - \frac{2}{1 - 2/3 z^{-1}} ; ROC: |z| > 2/3  $$


قطب سیستم $a = 2/3$ و صفر آن $z=2$ است.


\item
$$h[n] = 3\delta[n] - 2 (2/3)^n u[n]$$


\item

$$3z Y(z) -2 Y(z) = 3(zX(z) - 2 X(z))) \rightarrow$$
$$3 y[n+1] - 2y[n] = 3(x[n+1] - 2x[n])$$

\item

بله چون ROC تابع $H(z)$ آن شامل دایره واحد می شود.

\item

علی است. با توجه به این که ناحیه همگرایی $H(z)$ ناحیه بیرونی یک دایره است و شامل $\infty$ می شود.
\end{enumerate}

\section{سوال پنجم}


\begin{enumerate}[label = \Alph*)]
	\item
$$
\begin{array}{c}
	h[n]=\left\{\begin{array}{ll}
		a^{n}, & n \geq 0 \\
		0, & n<0
	\end{array}\right. \\
	x[n]=\left\{\begin{array}{lr}
		1, & 0 \leq n \leq N-1 \\
		0, & \text { otherwise }
	\end{array}\right.
\end{array}
$$

$$h[n] = a^n u[n] ~~~,~~~ x[n] = u[n] - u[n-N]$$

$$y[n] = x[n] \star h[n]$$
$$y[n] = \sum_{k=-\infty}^{\infty} x[n] h[n-k] = \sum_{k=-\infty}^{\infty} (u[k] - u[k-N])a^{n-k} u[n-k]=$$
$$\sum_{k=0}^{N-1} a^{n-k} u[n-k]$$

$$
=\left\{\begin{array}{ll}
	0 & n<0 \\
	\sum_{k=0}^{n} a^{n-k}=\sum_{k=0}^{n} a^{n}\left(\frac{1}{a}\right)^{k} & 0 \leq n \leq N-1 \\
	\sum_{k=0}^{N-1} a^{n-k}=\sum_{k=0}^{N-1} a^{n}\left(\frac{1}{a}\right)^{k} & n>N-1
\end{array}\right.
$$

$$
=\left\{\begin{array}{ll}
	0 & n<0 \\
	a^{n} \cdot \frac{1-\left(\frac{1}{a}\right)^{n+1}}{1-\left(\frac{1}{a}\right)}=a^{n} \cdot \frac{1-a^{-(n+1)}}{1-a^{-1}} & 0 \leq n \leq N-1 \\
	a^{n} \cdot \frac{1-\left(\frac{1}{a}\right)^{N}}{1-\left(\frac{1}{a}\right)}=a^{n} \cdot \frac{1-a^{-N}}{1-a^{-1}} & n>N-1
\end{array}\right.
$$

$$
y[n]=\frac{a^{n}-a^{-1}}{1-a^{-1}}(u[n]-u[n-N])+\frac{a^{n}-a^{(n-N)}}{1-a^{-1}} u[n-N]
$$

\item

$$
X(z)=\frac{1}{1-z^{-1}}-\frac{z^{-N}}{1-z^{-1}}=\frac{1-z^{-N}}{1-z^{-1}} \quad, \quad \text { ROC: }|z|>1
$$


$$
h[n]=a^{n} u[n] \rightarrow
H(z)=\frac{1}{1-a z^{-1}} \quad, \quad \text { ROC }:|z|>a
$$


$$Y(z) = \frac{1 - z^{-N}}{(1-z^{-1}) (1-az^{-1})} = (1 - z^{-N})(\frac{\frac{1}{1-a}}{1- z^{-1}} + \frac{\frac{1}{1 - a^{-1}}}{1 - a z^{-1}}) = (1 - z^{-N}) M(z)$$

$$m[n] = \frac{1}{1-a} u[n] + \frac{1}{1-a^{-1}} a^n u[n]$$

$$y[n] = m[n] - m[n-N] =(\frac{1}{1-a} u[n] + \frac{1}{1-a^{-1}} a^n u[n]) - (\frac{1}{1-a} u[n-N] + \frac{1}{1-a^{-1}} a^{n-N} u[n-N]) $$

و با حالت بندی داریم:

$$
y[n]=\left\{\begin{array}{ll}
	0 & n<0 \\
	a^{n} \cdot \frac{1-\left(\frac{1}{a}\right)^{n+1}}{1-\left(\frac{1}{a}\right)}=a^{n} \cdot \frac{1-a^{-(n+1)}}{1-a^{-1}} & 0 \leq n \leq N-1 \\
	a^{n} \cdot \frac{1-\left(\frac{1}{a}\right)^{N}}{1-\left(\frac{1}{a}\right)}=a^{n} \cdot \frac{1-a^{-N}}{1-a^{-1}} & n>N-1
\end{array}\right.
$$

پس جواب هر دو حالت یکسان است.

\end{enumerate}



\end{document}



