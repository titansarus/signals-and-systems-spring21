\documentclass[12pt]{article}
\usepackage{graphicx,import}
\usepackage[svgnames]{xcolor} 
\usepackage{fancyhdr}
\usepackage{subfig}
\usepackage{hyperref}
\usepackage{enumitem}
\usepackage{cite}
\usepackage[many]{tcolorbox}
\usepackage{listings }
\usepackage[a4paper, total={6in, 8in} , bottom = 25mm , top = 25mm, headheight = 1.25cm , includehead,includefoot,heightrounded ]{geometry}
\usepackage{afterpage}
\usepackage{amssymb}
\usepackage{pdflscape}
\usepackage{gensymb}
\usepackage{textcomp}
\usepackage{tikz,pgfplots}
\usepackage{xecolor}
\usepackage{rotating}
\usepackage{pdfpages}
\usepackage[Kashida]{xepersian}
\usepackage[T1]{fontenc}
\usepackage{tikz}
\usepackage[utf8]{inputenc}
\usepackage{PTSerif} 
\usepackage{seqsplit}

\usepackage[edges]{forest}

\usepackage{listings}
\usepackage{xcolor}

\hypersetup{
	colorlinks   = true, %Colours links instead of ugly boxes
	urlcolor     = blue, %Colour for external hyperlinks
	linkcolor    = blue, %Colour of internal links
	citecolor   = red %Colour of citations
}
 
\definecolor{codegreen}{rgb}{0,0.6,0}
\definecolor{codegray}{rgb}{0.5,0.5,0.5}
\definecolor{codepurple}{rgb}{0.58,0,0.82}
\definecolor{backcolour}{rgb}{0.95,0.95,0.92}
 
\NewDocumentCommand{\codeword}{v}{
\texttt{\textcolor{blue}{#1}}
}
\lstset{language=java,keywordstyle={\bfseries \color{blue}}}

\lstdefinestyle{mystyle}{
    backgroundcolor=\color{backcolour},   
    commentstyle=\color{codegreen},
    keywordstyle=\color{magenta},
    numberstyle=\tiny\color{codegray},
    stringstyle=\color{codepurple},
    basicstyle=\ttfamily\normalsize,
    breakatwhitespace=false,         
    breaklines=true,                 
    captionpos=b,                    
    keepspaces=true,                 
    numbers=left,                    
    numbersep=5pt,                  
    showspaces=false,                
    showstringspaces=false,
    showtabs=false,                  
    tabsize=2
}

\lstset{style=mystyle}

\settextfont[Scale=1.2 ,BoldFont={Bahij Nazanin-Bold.ttf} , ItalicFont = {IRNazaninIranic.ttf}]{Bahij Nazanin-Regular.ttf}
\setlatintextfont[Scale = 1.0]{Garamond}
\DefaultMathsDigits 
\DeclareMathSizes{11}{19}{13}{9} 
%\DeclareMathSizes{12}{14.4}{8}{9}





\newenvironment{changemargin}[2]{%
\begin{list}{}{%
\setlength{\topsep}{0pt}%
\setlength{\leftmargin}{#1}%
\setlength{\rightmargin}{#2}%
\setlength{\listparindent}{\parindent}%
\setlength{\itemindent}{\parindent}%
\setlength{\parsep}{\parskip}%
}%
\item[]}{\end{list}}


\definecolor{foldercolor}{RGB}{124,166,198}

\tikzset{pics/folder/.style={code={%
    \node[inner sep=0pt, minimum size=#1](-foldericon){};
    \node[folder style, inner sep=0pt, minimum width=0.3*#1, minimum height=0.6*#1, above right, xshift=0.05*#1] at (-foldericon.west){};
    \node[folder style, inner sep=0pt, minimum size=#1] at (-foldericon.center){};}
    },
    pics/folder/.default={20pt},
    folder style/.style={draw=foldercolor!80!black,top color=foldercolor!40,bottom color=foldercolor}
}

\forestset{is file/.style={edge path'/.expanded={%
        ([xshift=\forestregister{folder indent}]!u.parent anchor) |- (.child anchor)},
        inner sep=1pt},
    this folder size/.style={edge path'/.expanded={%
        ([xshift=\forestregister{folder indent}]!u.parent anchor) |- (.child anchor) pic[solid]{folder=#1}}, inner xsep=0.6*#1},
    folder tree indent/.style={before computing xy={l=#1}},
    folder icons/.style={folder, this folder size=#1, folder tree indent=3*#1},
    folder icons/.default={12pt},
}

\begin{document}


%%% title pages
\begin{titlepage}
\begin{center}
        
\vspace*{0.7cm}

\includegraphics[width=0.4\textwidth]{sharif1.png}\\
\vspace{0.5cm}
\textbf{ \Huge{\emph ‌سیگنال‌ها و سیستم‌ها} }\\
\vspace{0.5cm}
\textbf{ \Large{ تمرین دوم} }
\vspace{0.2cm}
       
 
      \large \textbf{دانشکده مهندسی کامپیوتر}\\\vspace{0.2cm}
    \large   دانشگاه صنعتی شریف\\\vspace{0.2cm}
       \large   ﻧﯿﻢ سال دوم 00-99 \\\vspace{0.2cm}
      \noindent\rule[1ex]{\linewidth}{1pt}
استاد:\\
    \textbf{{جناب آقای دکتر منظوری شلمانی}}


    \vspace{0.15cm}
نام و نام خانوادگی:\\

       
    \textbf{{امیرمهدی نامجو - 97107212}}
\end{center}
\end{titlepage}
%%% title pages


%%% header of pages
\newpage
\pagestyle{fancy}
\fancyhf{}
\fancyfoot{}
\cfoot{\thepage}
\chead{تمرین دوم}
\rhead{\includegraphics[width=0.1\textwidth]{sharif.png}}
\lhead{امیرمهدی نامجو}
%%% header of pages

\KashidaOff

\section{سوال اول}


\begin{enumerate}[label = \Alph*)]
	
	\item
	$x(t) = e^{3 |t|}$
	
	$$\mathcal{L}(x(t)) = \int_{-\infty}^{\infty} e^{3 |t|} e^{-s t} dt = \int_{0}^{\infty} e^{(3-s)t} dt+ \int_{-\infty}^{0} e^{(-3-s)t} dt$$
	
	$$= \frac{e^{(3-s)t}}{3-s}|_{0}^{\infty} + \frac{e^{(-3-s)t}}{-3-s}|_{-\infty}^{0} = \frac{1}{s-3} + \frac{-1}{s+3} = \frac{6}{s^2 - 9}$$
	
	با این حال باید توجه کرد که ناحیه همگرایی عامل اول برای $Re(s)>3$ بوده و ناحیه همگرایی عامل دوم جمع
	$-3 - Re(s) > 0 \rightarrow Re(s)<-3$
	است. در نتیجه این عبارت تبدیل لاپلاس ندارد چون ناحیه همگرایی کلی آن تهی است.
	
	\item
	$x(t) = e^{-3 |t|}$
	
		$$\mathcal{L}(x(t)) = \int_{-\infty}^{\infty} e^{-3 |t|} e^{-s t} dt = \int_{0}^{\infty} e^{(-3-s)t} dt+ \int_{-\infty}^{0} e^{(3-s)t} dt$$
	
	$$= \frac{e^{(-3-s)t}}{-3-s}|_{0}^{\infty} + \frac{e^{(3-s)t}}{3-s}|_{-\infty}^{0} = \frac{1}{s+3} + \frac{-1}{s-3} = \frac{6}{s^2 - 9}$$
	
	در این مورد ناحیه همگرایی عامل اول
	 $Re(s) >-3$
	 و برای عامل دوم
	 $Re(s)<3$
	 است که باعث می شود تبدیل لاپلاس درستی با ناحیه همگرایی
	 $-3 < Re(s) < 3$
	 داشته باشیم.
	 
	 
	 
	 \item
	 $$x(t) =e^{(-1 + j) t} \cos(3t) u(t)$$
	
	
	براساس جدول تبدیل لاپلاس:
	
	$$\mathcal{L}(x(t)) = \frac{s - (-1 +j)}{(s - (-1 +j))^2 + 9)} = \frac{s +1 - j}{9 + s^2 + (2-2j)s - 2j}$$
	
	برای ناحیه همگرایی عامل کسینوس صرفا نوسان ساز است و تاثیری ندارد. توان موهومی هم ایجاد کننده عوامل نوسان ساز است. تنها توان حقیقی مهم است. با توجه به Right-Side بودن سیگنال، ناحیه همگرایی
	$$Re(s)>-1$$
	است.
	
	
	
\end{enumerate}

\section{سوال دوم}


\begin{enumerate}[label = \Alph*)]
	
	\item
$$
\frac{s}{s^{2}+4}-\frac{5}{s+2}-\frac{1}{s-2} , 0 < Re(s) < 2
$$

برای عبات اول، معکوس آن $\cos(2t)$ خواهد بود. برای عبارت دوم معکوس آن
$5 e^{-2t}$
و برای عبارت سوم معکوس آن
$e^{2t}$
خواهد بود.

با توجه به ناحیه همگرایی متوجه می شویم که عبارت $\cos(2t)$ که ناحیه همگرایی مربوط به صفر را ایجاد کرده باید Right-Sided باشد و در نتیجه $e^{2t}$ هم Right-Sided خواهد بود. اما عبارت $e^{-2t}$ به صورت Left-Sided خواهد بود.


	$$\mathcal{L}^{-1} (X(s))= \cos(2t) u(t) - 5 e^{-2t} u(t) - (- e^{2t} u(-t)) = \cos(2t) u(t) - 5 e^{-2t} u(t) + e^{2t} u(-t)$$


	
	\item
	
	$$
	X(s) =\frac{s+2}{s^{2}+7 s+12} , -4<Re(s)<-3
	$$
	
	$$
	\frac{s+2}{s^{2}+7 s+12} = \frac{2}{s+4} - \frac{1}{s+3}
	$$
	
	
	عامل اصلی تبدیل لاپلاس اولی $e^{-4t}$ و دومی $e^{-3t}$ است. با توجه به ناحیه همگرایی داده شده، برای $-4$ باید Right-Sided داشته باشیم و برای $-3$ عبارت Left-Sided پس:
	
	$$\mathcal{L}^{-1} (X(s))= 2 e^{-4t} u(t) - (- e^{-3t}u(-t)) = 2 e^{-4t} u(t) + e^{-3t}u(-t)$$
	
\end{enumerate}


\section{سوال سوم}

با توجه به شکل داده شده و این که صفر نداریم، یعنی صورت عبارت تبدیل لاپلاس یک عدد ثابت است.  از طرفی با توجه به نقاط قطب ها، عامل مختلط $s^2-2s+2$ و عوامل $s+2$ و $s+1$ وجود دارند. یعنی عبارت اصلی به شکل زیر است:

$$\frac{a}{(s^2-2s+2)(s+2)(s+1)}$$
است. عبارت اول مخرج براساس
$(s - (1+j))(s-(1-j))$
بدست آمده است.

این عبارت را اگر تبدیل به کسر های جزئی کنیم به عبارت زیر می رسیم:

$$X(S)=\frac{a}{10}(\frac{2}{s+1} + \frac{-1}{s+2} + \frac{2  }{(s-1)^2 + 1} + \frac{- s}{(s-1)^2 +1})$$
$$= \frac{a}{10}(\frac{2}{s+1} + \frac{-1}{s+2} + \frac{1  }{(s-1)^2 + 1} + \frac{- (s-1)}{(s-1)^2 +1})$$

با توجه به ناحیه همگرایی داده شده، یعنی عبارت های سینوسی و کسینوسی که از دو بخش آخر بدست می آیند هر دو باید Left-Sided باشند زیرا بخش حقیقی قطب آن ها $1$ است و ناحیه همگرایی در سمت چپ آن اتفاق افتاده است. عبارت مربوط به $s+1$ هم باید Left-Sided باشد و عبارت مربوط به $s+2$ باید Right-Sided باشد.

با توجه به این مسائل و طبق جدول تبدیل لاپلاس داریم:

$$\mathcal{L}^{-1} (X(s)))=\frac{a}{10}\left((-2 e^{-t} u(-t)) + (- e^{-2t}u(t)) + (-e^t \sin(t) u(-t)) + (e^t \cos(t) u(-t))\right)$$

\end{document}



