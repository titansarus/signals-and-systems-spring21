\documentclass[12pt]{article}
\usepackage{graphicx,import}
\usepackage[svgnames]{xcolor} 
\usepackage{fancyhdr}
\usepackage{subfig}
\usepackage{hyperref}
\usepackage{enumitem}
\usepackage{cite}
\usepackage{fancyvrb}
\usepackage[many]{tcolorbox}
\usepackage{mathtools}
\usepackage{listings }
\usepackage[a4paper, total={6in, 8in} , bottom = 25mm , top = 25mm, headheight = 1.25cm , includehead,includefoot,heightrounded ]{geometry}
\usepackage{afterpage}
\usepackage{amssymb}
\usepackage{pdflscape}
\usepackage{gensymb}
\usepackage{textcomp}
\usepackage{tikz,pgfplots}
\usepackage{xecolor}
\usepackage{rotating}
\usepackage{pdfpages}
\usepackage[Kashida]{xepersian}
\usepackage[T1]{fontenc}
\usepackage{tikz}
\usepackage[utf8]{inputenc}
\usepackage{PTSerif} 
\usepackage{seqsplit}

\usepackage[edges]{forest}

\usepackage{listings}
\usepackage{xcolor}

\hypersetup{
	colorlinks   = true, %Colours links instead of ugly boxes
	urlcolor     = blue, %Colour for external hyperlinks
	linkcolor    = blue, %Colour of internal links
	citecolor   = red %Colour of citations
}
 
\definecolor{codegreen}{rgb}{0,0.6,0}
\definecolor{codegray}{rgb}{0.5,0.5,0.5}
\definecolor{codepurple}{rgb}{0.58,0,0.82}
\definecolor{backcolour}{rgb}{0.95,0.95,0.92}
 
\NewDocumentCommand{\codeword}{v}{
\texttt{\textcolor{blue}{#1}}
}
\lstset{language=java,keywordstyle={\bfseries \color{blue}}}

\lstdefinestyle{mystyle}{
    backgroundcolor=\color{backcolour},   
    commentstyle=\color{codegreen},
    keywordstyle=\color{magenta},
    numberstyle=\tiny\color{codegray},
    stringstyle=\color{codepurple},
    basicstyle=\ttfamily\normalsize,
    breakatwhitespace=false,         
    breaklines=true,                 
    captionpos=b,                    
    keepspaces=true,                 
    numbers=left,                    
    numbersep=5pt,                  
    showspaces=false,                
    showstringspaces=false,
    showtabs=false,                  
    tabsize=2
}

\lstset{style=mystyle}

\settextfont[Scale=1.2 ,BoldFont={Bahij Nazanin-Bold.ttf} , ItalicFont = {IRNazaninIranic.ttf}]{Bahij Nazanin-Regular.ttf}
\setlatintextfont[Scale = 1.0]{Garamond}
\DefaultMathsDigits 
\DeclareMathSizes{11}{19}{13}{9} 
%\DeclareMathSizes{12}{14.4}{8}{9}





\newenvironment{changemargin}[2]{%
\begin{list}{}{%
\setlength{\topsep}{0pt}%
\setlength{\leftmargin}{#1}%
\setlength{\rightmargin}{#2}%
\setlength{\listparindent}{\parindent}%
\setlength{\itemindent}{\parindent}%
\setlength{\parsep}{\parskip}%
}%
\item[]}{\end{list}}


\definecolor{foldercolor}{RGB}{124,166,198}

\tikzset{pics/folder/.style={code={%
    \node[inner sep=0pt, minimum size=#1](-foldericon){};
    \node[folder style, inner sep=0pt, minimum width=0.3*#1, minimum height=0.6*#1, above right, xshift=0.05*#1] at (-foldericon.west){};
    \node[folder style, inner sep=0pt, minimum size=#1] at (-foldericon.center){};}
    },
    pics/folder/.default={20pt},
    folder style/.style={draw=foldercolor!80!black,top color=foldercolor!40,bottom color=foldercolor}
}

\forestset{is file/.style={edge path'/.expanded={%
        ([xshift=\forestregister{folder indent}]!u.parent anchor) |- (.child anchor)},
        inner sep=1pt},
    this folder size/.style={edge path'/.expanded={%
        ([xshift=\forestregister{folder indent}]!u.parent anchor) |- (.child anchor) pic[solid]{folder=#1}}, inner xsep=0.6*#1},
    folder tree indent/.style={before computing xy={l=#1}},
    folder icons/.style={folder, this folder size=#1, folder tree indent=3*#1},
    folder icons/.default={12pt},
}

\begin{document}


%%% title pages
\begin{titlepage}
\begin{center}
        
\vspace*{0.7cm}

\includegraphics[width=0.4\textwidth]{sharif1.png}\\
\vspace{0.5cm}
\textbf{ \Huge{\emph ‌سیگنال‌ها و سیستم‌ها} }\\
\vspace{0.5cm}
\textbf{ \Large{ تمرین پنجم} }
\vspace{0.2cm}
       
 
      \large \textbf{دانشکده مهندسی کامپیوتر}\\\vspace{0.2cm}
    \large   دانشگاه صنعتی شریف\\\vspace{0.2cm}
       \large   ﻧﯿﻢ سال دوم 00-99 \\\vspace{0.2cm}
      \noindent\rule[1ex]{\linewidth}{1pt}
استاد:\\
    \textbf{{جناب آقای دکتر منظوری شلمانی}}


    \vspace{0.15cm}
نام و نام خانوادگی:\\

       
    \textbf{{امیرمهدی نامجو - 97107212}}
\end{center}
\end{titlepage}
%%% title pages


%%% header of pages
\newpage
\pagestyle{fancy}
\fancyhf{}
\fancyfoot{}
\cfoot{\thepage}
\chead{تمرین پنجم}
\rhead{\includegraphics[width=0.1\textwidth]{sharif.png}}
\lhead{امیرمهدی نامجو}
%%% header of pages

\KashidaOff

\section{سری فوریه}
\subsection{سوال اول}

\subsubsection{بخش a}


$$
f(x)=\left\{\begin{array}{lr}
	\pi-x & 0 \leq x \leq \pi \\
	x-\pi & -\pi \leq x \leq 0
\end{array}\right.
$$

ابتدا عامل DC را بدست می‌آوریم:

$$a_0 = \frac{1}{2\pi} \int f(x) dx = \frac{1}{2\pi} \int_{-\pi}^{0} (x- \pi) dx + \int_{0}^{\pi} (\pi -x) dx$$
$$= \frac{1}{2\pi} (\frac{-3\pi^2}{2} - \frac{\pi^2}{2}) = \boxed{- \pi} $$

برای ضرایب کسینوسی داریم:

$$a_n = \frac{1}{\pi} \int (f(x) \cos (nx)) dx$$

$$= \frac{1}{\pi} (\int_{-\pi}^{0} (x - \pi) \cos (nx) dx + \int_{0}^{\pi} (\pi - x) \cos (nx) dx)$$

ابتدا لازم است اشاره کنیم که
$$\int x \cos (nx) = \frac{1}{n^2} \cos(nx) + \frac{1}{n} x \sin (nx)$$

با توجه به این موضوع، از عبارت بالا می‌توان به راحتی انتگرال گرفت:


$$= \frac{1}{\pi}( (\frac{\cos (n x)}{n^2}+\frac{x \sin (n x)}{n}-\frac{\pi  \sin (n x)}{n}) |_{-\pi}^{0} + (-\frac{\cos (n x)}{n^2}+\frac{\pi  \sin (n x)}{n}-\frac{x \sin (n x)}{n})|_{0}^{\infty})$$

بعد از ساده سازی و محاسبات داریم:

$$\boxed{a_n = -\frac{2 (\pi  n \sin (\pi  n)+\cos (\pi  n)-1)}{\pi  n^2} = \frac{-2((-1)^n -1)}{\pi n^2}}$$


برای محاسبات ضریب سینوسی داریم:

$$b_n = \frac{1}{\pi} \int (f(x) sin(nx)) dx$$

$$= \frac{1}{\pi} (\int_{-\pi}^{0} (x - \pi) \sin (nx) dx + \int_{0}^{\pi} (\pi - x) \sin (nx) dx)$$


ابتدا لازم است اشاره کنیم که
$$\int x \sin(nx) = \frac{\sin (n x)}{n^2}-\frac{x \cos (n x)}{n}  $$

با توجه به این موضوع، عبارت بالا مانند بخش قبل به راحتی قابل محاسبه است. جوابی که در نهایت به آن می رسیم به صورت زیر است:


$$\boxed{b_n = \frac{2-2 \cos (\pi  n)}{n} = \frac{2 - 2 (-1)^n}{n}}$$


و جواب نهایی به صورت:

$$f(x) = a_0 + \sum_{n=1}^{N} (a_n \cos (n x) + b_n \sin (n x))$$

خواهد بود. (تقسیم بر ۲ فرمول $a_0$ را به نوعی در خود انتگرال آن تاثیر داده‌ام)

کد آن در فایل 
\lr{\Verb+P1\_Q1\_a.py+}
قرار دارد. نمودار در صفحه بعد قرار گرفته است. شکل بالایی خود تابع و شکل های بعدی به ازای $N=2,5,20,50$ هستند.

\begin{center}
	\includegraphics[width = 1.0 \textwidth]{images/1.pdf}
\end{center}


\newpage

\subsubsection{بخش b}

$$
f(x)=\left\{\begin{array}{lr}
	1 & 0 \leq x<\frac{\pi}{2} \\
	0 & \frac{\pi}{2} \leq x<\pi \\
	0 & -\pi \leq x<0
\end{array}\right.
$$

$$a_0 = \frac{1}{2\pi} \int f(x) dx = \frac{1}{2\pi} \int_{0}^{\pi/2} 1 dx = \boxed{ \frac{1}{4}}$$


$$a_n = \frac{1}{\pi} \int (f(x) \cos (nx)) dx$$

$$=\frac{1}{\pi}\int_{0}^{\pi/2} \cos (nx) dx = \frac{1}{\pi }\frac{\sin(nx)}{n}|_{0}^{\pi/2}  = \boxed{\frac{\sin \left(\frac{\pi  n}{2}\right)}{\pi  n}}$$

$$b_n = \frac{1}{\pi} \int (f(x) \sin (nx)) dx$$
$$= \frac{1}{\pi} \int_{0}^{\pi/2} \sin (nx) dx = - \frac{1}{\pi} \frac{\cos (n x)}{n} |_{0}^{\pi/2}  = \boxed{\frac{2 \sin ^2\left(\frac{\pi  n}{4}\right)}{\pi  n}}$$

که در بالا از اتحاد
$\cos(2 \theta) = 1 - 2\sin^2 (\theta)$
استفاده شده است.


و جواب نهایی به صورت:

$$f(x) = a_0 + \sum_{n=1}^{N} (a_n \cos (n x) + b_n \sin (n x))$$



کد آن در فایل 
\lr{\Verb+P1\_Q1\_b.py+}
قرار دارد. نمودار در صفحه بعد قرار گرفته است. شکل بالایی خود تابع و شکل های بعدی به ازای $N=2,5,20,50$ هستند.

\begin{center}
	\includegraphics[width = 1.0 \textwidth]{images/2.pdf}
\end{center}

\newpage
\subsection{سوال دوم}
\subsubsection{بخش a}

$$\cos(4t) = \frac{1}{2} e^{-4 j t}+\frac{1}{2} e^{4 j t} $$

$$\sin(6t) = \frac{1}{2j} e^{6 j t}-\frac{1}{2j} e^{-6 j t} $$

در نتیجه ضرایب سری فوریه برای 
$\cos(4t)+\sin(6t)$
به صورت زیر است:

$$a_4 = \frac{1}{2} , a_{-4} = \frac{1}{2} , a_{6} = \frac{1}{2 j} , a_{-6} = \frac{-1}{2j}$$
و به ازای $k\neq \pm4,\pm6$ داریم $a_k = 0$

\subsubsection{بخش b}
\begin{center}
	\includegraphics[width = 0.5 \textwidth]{images/3.png}
\end{center}


$$\omega_0 = \frac{2\pi}{T}$$

$$a_k = \frac{1}{T_0} \int_{-T_0/2}^{T_0/2} x(t) e^{-j k \frac{2\pi}{T_0} t} dt$$

برای $k=0$ به طور جداگانه محاسبه کرده و داریم:

$$\boxed{a_0 = \frac{1}{T_0} \int_{-T_0/2}^{T_0/2} x(t) dt = 0}$$

برای باقی موارد داریم:

$$a_k = \frac{1}{T_0} \int_{-T_0/2}^{T_0/2} x(t) e^{-j k \frac{2\pi}{T_0} t} dt = \frac{1}{T_0}(\int_{-T_0/2}^{0} (-A) e^{-j k \frac{2\pi}{T_0} t} dt + \int_{0}^{T_0/2} (A) e^{-j k \frac{2\pi}{T_0} t} dt )$$

$$= \frac{1}{T_0} (\frac{A T_0 j (-1 + e^{j k \pi})}{2 k \pi} + \frac{- A  T_0 j (1 - e^{j k \pi})}{2 k \pi})$$

$$\boxed{= \frac{A j e^{-j k \pi} (-1 + e^{j k \pi})^2}{2 k \pi}}$$


\subsubsection{بخش c}

دوره تناوب پایه $|\sin (x)|$ برابر $\pi$ است و عملا مانند $\sin$ مثبتی بین $0$ تا $\pi$ است که در همه تناوب‌هایش تکرار می‌شود. در نتیجه باید براساس این تناوب حل کرد.

$$a_k = \frac{1}{\pi} \int_{0}^{\pi} |\sin (x)| e^{- 2 j k x} dx = \frac{1}{\pi}\int_{0}^{\pi} \sin (x) e^{- 2 j k x} dx $$

برای ضریب $a_0$ داریم:

$$a_0 = \frac{1}{\pi} \int_{0}^{\pi} \sin (x) dx = \frac{2}{\pi}$$


برای سایر ضرایب داریم:

$$a_k = \frac{1}{\pi}\int_{0}^{\pi} \frac{1}{2 j} (e^{i x} - e^{-ix}) e^{- 2 j k x}  dx$$

$$=\frac{1}{2 \pi} \left(\frac{e^{-j (2 k-1) x}}{2 k-1}-\frac{e^{-j (2 k+1) x}}{2 k+1}\right)|_{0}^{\pi}$$

$$=\frac{1}{2\pi} \left(\frac{1}{2 k+1}-\frac{1}{2 k-1}\right)+\frac{1}{2\pi} \left(\frac{e^{-j \pi  (2 k-1)}}{2 k-1}-\frac{e^{-j \pi  (2 k+1)}}{2 k+1}\right)$$

$$=\boxed{\frac{1}{\pi}\frac{1+e^{-2 j \pi  k}}{1-4 k^2}}$$

\subsection{سوال سوم}

\subsubsection{بخش a}



$$x(t) = 
+-2j e^{-2j \omega_0 t}
+-1j e^{-1j \omega_0 t}
+1j e^{1j \omega_0 t}
+2j e^{2j \omega_0 t}
$$

$$=
-\frac{4}{2j}(e^{2j \omega_0 t} - e^{-2j \omega_0 t})
-\frac{2}{2j}(e^{j \omega_0 t} - e^{-j \omega_0 t})$$

$$\boxed{= -4 \sin(2 \omega_0 t) -2 \sin( \omega_0 t)}$$


\subsubsection{بخش b}

عبارت مورد نظر باید ما را به یاد سری فوریه قطار ضربه بیندازد.

$$\delta_{T_0} (t) = \sum_{k=-\infty}^{\infty} c_k e^{jk\omega_0 t}$$

برای ضرایب فوریه چنین چیزی داریم:
$$
c_{k}=\frac{1}{T_{0}} \int_{-T_{0} / 2}^{T_{0} / 2} \delta(t) e^{-j k \omega_{0} t} d t=\frac{1}{T_{0}}
$$

$$
\delta_{T_{0}}(t)=\sum_{k=-\infty}^{\infty} \delta\left(t-k T_{0}\right)=\frac{1}{T_{0}} \sum_{k=-\infty}^{\infty} e^{j k \omega_{0} t} \quad \omega_{0}=\frac{2 \pi}{T_{0}}
$$


با توجه به این موضوع برای چیزی که در صورت سوال داده شده، می‌توانیم آن را معادل با 

$$
z(t)=\sum_{k=-\infty}^{\infty} e^{j k \omega_{0} t} \delta(t-T_0 k+2 k)
$$

بدانیم.

در عبارت بالا $2k$ برای زوج سازی و سپس $e$ برای شیفت فرکانسی اضافه شده است که باعث بشود که تنها عبارت‌های فرد $1$ بمانند و عبارت‌های زوج $0$ شوند.



\newpage

\subsection{سوال چهارم}

در سوال نمادهای $e_k$ و $d_k$ استفاده شده است ولی برای راحتی کار و از آن جایی که کلا دو سیگنال اصلی داریم، از $a_k$ و $b_k$ در جواب استفاده شده است.

$$
x_1[n] x_2[n]=\sum_{k=0}^{N_0-1} \sum_{l=0}^{N_0-1} a_{k} b_{l} e^{j(2 \pi / N_0)(k+l) n}
$$

$$
x_1[n] x_2[n]=\sum_{k=0}^{(N_0-1)} \sum_{l^{\prime}=k}^{(k+N_0-1)} a_{k} b_{l^{\prime}-k} e^{j(2 \pi / N_0)^{\prime} n}
$$

با توجه به متناوب بودن
$b_{l' -k}$
و
$e^{j 2\pi /N_0 l' n}$
داریم:

$$
x_1[n] x_2[n]=\sum_{k=0}^{N_0-1} \sum_{l^{\prime}=0}^{N_0-1} a_{k} b_{l^{\prime}-k} e^{j(2 \pi / N_0) t^{\prime} n}=\sum_{l=0}^{N_0-1}\left[\sum_{k=0}^{N_0-1} a_{k} b_{l-k}\right] e^{j(2 \pi / N_0) l n}
$$

پس

$$
c_{k}=\sum_{t=0}^{N_0-1} a_{k} b_{l-k}
$$

و معادلا:
$$
c_{k}=\sum_{k=0}^{N_0-1} b_{k} a_{l-k}
$$

برای اثبات رابطه پارسوال داریم:


$$
N_0 \sum_{l=\langle N_0\rangle} a_{l} b_{k-l}=\sum_{\langle N_0\rangle} x_1[n] x_2[n] e^{-j(2 \pi / N_0) k n}
$$

با قرار دادن $k=0$ داریم:

$$
N_0 \sum_{l=\langle N_0\rangle} a_{l} b_{-1}=\sum_{n= \langle N_0\rangle} x_1[n] x_2[n]
$$

در نتیجه:

$$
\frac{1}{N_{0}} \sum_{n=0}^{N_{0}-1} x[n]=\sum_{k=0}^{N_{0}-1} a_{k} b_{-k}
$$

\newpage
\subsection{سوال پنجم}

در نتیجه سوال قبل قرار می‌دهیم:

$$x_2[n] = x^*_1[n]$$

در نتیجه این موضوع داریم:

$$b_k = a^*_{-k}$$

پس

$$
\frac{1}{N_{0}} \sum_{n=0}^{N_{0}-1} x[n]=\sum_{n=0}^{N_{0}-1} a_{k} b_{-k}
$$

$$
\frac{1}{N_{0}} \sum_{n=0}^{N_{0}-1} x_1[n] x_1^*[n]=\sum_{k=0}^{N_{0}-1} a_{k} a^*_{k}
$$

بنابراین:

$$
\sum_{k=\langle n_0\rangle}\left|a_{k}\right|^{2}=\frac{1}{N_0} \sum_{n=\langle N_0\rangle}|x[n]|^{2} .
$$

\newpage
\subsection{سوال ششم}
$$
f(t)=\left\{\begin{array}{lc}
	t+\frac{5}{3} & -1 \leq t<0 \\
	-t+\frac{5}{3} & 0 \leq t<2 \\
	0 & 2 \leq t<4
\end{array}\right.
$$

\subsubsection{بخش a}
$$a_0 = \frac{1}{5} \int f(t) dt = \frac{1}{5}( \int_{-1}^{0} t + 5/3 dt  + \int_{0}^{2} -t + 5/3 dt )$$
$$= \frac{1}{5}(\frac{7}{6} + \frac{4}{3}) =\boxed{ \frac{1}{2}}$$

$$a_k = \frac{1}{5} \int_{-1}^{5} f(t) e^{-j k \frac{2\pi}{5} t} dt = \frac{1}{5} (\int_{-1}^{0} (t+\frac{5}{3}) e^{-j k \frac{2\pi}{5} t} dt + \int_{0}^{2} (-t+\frac{5}{3}) e^{-j k \frac{2\pi}{5} t} dt)$$


$$\frac{1}{5} (\frac{50 i \pi  k+e^{\frac{2 i \pi  k}{5}} (-75-20 i \pi  k)+75}{12 \pi ^2 k^2} + \frac{-50 i \pi  k+e^{\frac{-4}{5} i \pi  k} (-75-10 i \pi  k)+75}{12 \pi ^2 k^2})$$$$ = \frac{e^{\frac{1}{5} (-4) i \pi  k} \left(-2 i \pi  k+30 e^{\frac{4 i \pi  k}{5}}+e^{\frac{6 i \pi  k}{5}} (-15-4 i \pi  k)-15\right)}{12 \pi ^2 k^2} $$


یا اگر روش فرمول کسینوس و سینوس را برویم داریم:


$$a_n = \frac{2}{5} \int (f(t) \cos (\frac{2\pi}{5} n t)) dt$$
$$ =\frac{2}{5} (\int_{-1}^{0} (t+5/3) \cos(\frac{2 \pi}{5} n t) dt + \int_{0}^{2} (-t + 5/3) \cos (\frac{2 \pi}{5} n t) dt)$$
$$= \frac{\sin ^2\left(\frac{\pi  n}{5}\right) \left(4 \pi  n \sin \left(\frac{2 \pi  n}{5}\right)+30 \cos \left(\frac{2 \pi  n}{5}\right)+45\right)}{3 \pi ^2 n^2}$$


$$b_n = \frac{2}{5} (\int (f(t) \sin (\frac{2\pi}{5} n t)) dt$$
$$ =\frac{2}{5}( \int_{-1}^{0} (t+5/3) \sin(\frac{2 \pi}{5} n t) dt + \int_{0}^{2} (-t + 5/3) \sin (\frac{2 \pi}{5} n t) dt)$$
$$= \frac{15 \left(\sin \left(\frac{2 \pi  n}{5}\right)-\sin \left(\frac{4 \pi  n}{5}\right)\right)+4 \pi  n \cos \left(\frac{2 \pi  n}{5}\right)+2 \pi  n \cos \left(\frac{4 \pi  n}{5}\right)}{6 \pi ^2 n^2}$$



$$f(x) = a_0 + \sum_{n=1}^{N} (a_n \cos (n x) + b_n \sin (n x))$$




\subsubsection{بخش b}

کدهای مسئله به زبان پایتون در فایل 
\lr{\Verb+P1\_Q6\_b.py+}
موجود است و جواب قسمت‌های بعد براساس آن تولید شده است:

جملات مد نظر در ادامه نوشته شده اند. توجه کنید که به دلیل ویژگی‌های اعداد Floating-Point عموما ضرایبی که صفر بوده‌اند به صورت عددی ضربدر $10^{-33}$ نوشته شده‌اند.

پس از آن ابتدا در یک شکل سیگنال‌ها به ازای مقادیر $N$ به صورت جداگانه رسم شده‌اند. سپس در اشکال بعدی، به ازای هر کدام از مقادیر، نمودار آن با رنگ نارنجی روی نمودار اصلی با رنگ آبی رسم شده است.



$$a_{ 1 } = 0.772711906482877 , b_{ 1 } = 0.07175375986881644$$

$$a_{ 2 } = 0.27113541693643917 , b_{ 2 } = 0.028002784979373037$$

$$a_{ 3 } = -0.004852111911900812 , b_{ 3 } = -0.08960735823011712$$

$$a_{ 4 } = 0.004714960180721709 , b_{ 4 } = -0.010817086948420714$$

$$a_{ 5 } = 1.5195743635847465e-33 , b_{ 5 } = 0.06366197723675814$$

$$a_{ 6 } = 0.04083290139095018 , b_{ 6 } = -0.0008212742065860439$$

$$a_{ 7 } = 0.04515821107548206 , b_{ 7 } = -0.011886611106858576$$

$$a_{ 8 } = -0.01831062003374954 , b_{ 8 } = -0.023451895441009625$$

$$a_{ 9 } = -0.007676952848145469 , b_{ 9 } = -0.00338756817877342$$

$$a_{ 10 } = 1.5195743635847456e-33 , b_{ 10 } = 0.03183098861837907$$

$$a_{ 11 } = 0.017911214823682062 , b_{ 11 } = -0.0010816983697769051$$

$$a_{ 12 } = 0.023201132067024215 , b_{ 12 } = -0.008867354363816892$$

$$a_{ 13 } = -0.013610002113093692 , b_{ 13 } = -0.012990392854308745$$

$$a_{ 14 } = -0.006730131246383232 , b_{ 14 } = -0.0019169012948294058$$

$$a_{ 15 } = 1.5195743635847434e-33 , b_{ 15 } = 0.021220659078919374$$

$$a_{ 16 } = 0.01118956850490962 , b_{ 16 } = -0.000907051304878608$$

$$a_{ 17 } = 0.015464268832541013 , b_{ 17 } = -0.006821294507069879$$

$$a_{ 18 } = -0.010581176024894123 , b_{ 18 } = -0.008919232952258454$$

$$a_{ 19 } = -0.005585535368095415 , b_{ 19 } = -0.0013214190722783403$$

$$a_{ 20 } = 1.5195743635847419e-33 , b_{ 20 } = 0.015915494309189534$$

$$a_{ 21 } = 0.008076648709406326 , b_{ 21 } = -0.0007562919849315991$$

$$a_{ 22 } = 0.011564843734622392 , b_{ 22 } = -0.005507870241109013$$

$$a_{ 23 } = -0.008613443680564381 , b_{ 23 } = -0.006775588961597922$$

$$a_{ 24 } = -0.004711199325475308 , b_{ 24 } = -0.0010040831881916652$$

$$a_{ 25 } = 1.5195743635847393e-33 , b_{ 25 } = 0.012732395447351627$$

$$a_{ 26 } = 0.006300406249170883 , b_{ 26 } = -0.0006432609418058376$$

$$a_{ 27 } = 0.009225781951316991 , b_{ 27 } = -0.004609416023114295$$

$$a_{ 28 } = -0.007250921405575788 , b_{ 28 } = -0.005457578628620997$$

$$a_{ 29 } = -0.004055794714702865 , b_{ 29 } = -0.0008081706869254678$$

$$a_{ 30 } = 1.519574363584735e-33 , b_{ 30 } = 0.010610329539459687$$

$$a_{ 31 } = 0.005157489248731366 , b_{ 31 } = -0.0005579230259798711$$

$$a_{ 32 } = 0.007669732145959915 , b_{ 32 } = -0.003959686879886606$$

$$a_{ 33 } = -0.006256136875045252 , b_{ 33 } = -0.004566755490351168$$

$$a_{ 34 } = -0.003553802680498653 , b_{ 34 } = -0.0006755979057181989$$

$$a_{ 35 } = 1.5195743635847316e-33 , b_{ 35 } = 0.009094568176679736$$

$$a_{ 36 } = 0.004362360888597353 , b_{ 36 } = -0.0004918855395621645$$

$$a_{ 37 } = 0.006561005399865466 , b_{ 37 } = -0.0034690828700930944$$

$$a_{ 38 } = -0.005499407016535415 , b_{ 38 } = -0.0039249666296024815$$

$$a_{ 39 } = -0.003159413834482431 , b_{ 39 } = -0.0005800860028377107$$

$$a_{ 40 } = 1.5195743635847268e-33 , b_{ 40 } = 0.007957747154594767$$

$$a_{ 41 } = 0.003778044173238204 , b_{ 41 } = -0.00043950225495442116$$

$$a_{ 42 } = 0.005731421240004929 , b_{ 42 } = -0.003085957785323703$$

$$a_{ 43 } = -0.004905005087522983 , b_{ 43 } = -0.0034408366129256586$$

$$a_{ 44 } = -0.0028423248893797424 , b_{ 44 } = -0.0005080735752180709$$

$$a_{ 45 } = 1.5195743635847215e-33 , b_{ 45 } = 0.00707355302630646$$

$$a_{ 46 } = 0.0033308907366752464 , b_{ 46 } = -0.00039703351695859266$$

$$a_{ 47 } = 0.0050875680137263376 , b_{ 47 } = -0.0027786711713284526$$

$$a_{ 48 } = -0.004426026511190478 , b_{ 48 } = -0.003062743915561243$$

$$a_{ 49 } = -0.0025822630276678576 , b_{ 49 } = -0.0004518742494494454$$

$$a_{ 50 } = 1.5195743635847159e-33 , b_{ 50 } = 0.006366197723675813$$


\begin{center}
	\includegraphics[width = 1.0 \textwidth]{images/6-1.pdf}
\end{center}



\begin{center}
	\includegraphics[width = 1.0 \textwidth]{images/6-2.pdf}
\end{center}

\subsubsection{بخش c}

کد این بخش در فایل
\lr{\Verb+P1\_Q6\_c+}
قرار دارد.



\begin{center}
	\includegraphics[width = 1.0 \textwidth]{images/6-3.pdf}
\end{center}

\newpage
\section{تبدیل فوریه}

\subsection{سوال اول}

$$X(jw) = \int_{-\infty}^{\infty} x(t) e^{-j \omega t} dt$$

\subsubsection{بخش a}

$$e^{- a |t|} \sin \omega_0 t$$

$$X(jw) = \int_{-\infty}^{\infty} e^{- a |t|} \sin (\omega_0 t) e^{-j \omega t}  dt$$
$$\int_0^{\infty}  \sin (\omega_0 t) e^{(-jw -a) t} + \int_{-\infty}^{0}  \sin (\omega_0 t) e^{(-jw + a)t}$$

$$= \int_{0}^{\infty} e^{-at} \sin(\omega_0 t) (e^{-jwt} - e^{jwt})$$

$$= -2j \int_{0}^{\infty} e^{at} \sin(\omega_0 t) \sin(\omega t)$$


$$= j \int_{0}^{\infty} e^{at} (\cos ((\omega_0 + \omega) t) - \cos((\omega_0 - \omega )t))$$

$$= j(e^{a t} \left(\frac{a \cos (t (\omega_0-\omega))+(\omega_0-\omega) \sin (t (\omega_0-\omega))}{a^2+(\omega_0-\omega)^2}-\frac{a \cos (t (\omega_0+\omega))+(\omega_0+\omega) \sin (t (\omega_0+\omega))}{a^2+(\omega_0+\omega)^2}\right))|_{0}^{\infty}$$

با شرط $a<0$ داریم:

$$= \frac{4 a \omega_0 \omega j}{\left(a^2+\omega_0^2\right)^2+2 \omega^2 (a-\omega_0) (a+\omega_0)+\omega^4}$$

\newpage


\subsubsection{بخش b}

$$
X(j \omega)=\int_{-1}^{1}(1+\cos (\pi t)) e^{-j \omega t} d t
$$

$$
\begin{gathered}
	X(j \omega)=\int_{-1}^{1} e^{-j \omega t} d t+\int_{-1}^{1} \frac{e^{j \pi t}+e^{-j \pi t}}{2} e^{-j \omega t} d t \\
	X(j \omega)=\left.\frac{e^{-j \omega t}}{-j \omega}\right|_{-1} ^{1}+\left.\frac{1}{2}\left(\frac{e^{j(\pi-\omega)}}{j(\pi-\omega)}+\frac{e^{-j(\pi+\omega) t}}{-j(\pi+\omega)}\right)\right|_{-1} ^{1} \\
	X(j \omega)=\frac{e^{-j \omega}-e^{j \omega}}{-j \omega}+\frac{1}{2}\left(\frac{e^{j(\pi-\omega)}-e^{j(\pi-\omega)}}{j(\pi-\omega)}+\frac{e^{-j(\pi+\omega)}-e^{j(\pi+\omega)}}{-j(\pi+\omega)}\right) \\
	X(j \omega)=\frac{2}{\omega} \cdot \frac{e^{j \omega}-e^{-j \omega}}{2 j}+\frac{1}{\pi-\omega} \cdot \frac{e^{j(\pi-\omega)}-e^{j(\pi-\omega)}}{2 j}+\frac{1}{\pi+\omega} \cdot \frac{e^{j(\pi+\omega)}-e^{-j(\pi+\omega)}}{2 j} \\
	X(j \omega)=\frac{2 \sin \omega}{\omega}+\frac{\sin (\pi-\omega)}{\pi-\omega}+\frac{\sin (\pi+\omega)}{\pi+\omega}
\end{gathered}
$$

$$\boxed{X(j\omega) = \frac{2 \sin \omega}{\omega}+\frac{\sin (\pi-\omega)}{\pi-\omega}+\frac{\sin (\pi+\omega)}{\pi+\omega}}$$

\subsubsection{بخش c}


می‌دانیم تبدیل فوریه $e^{a|t|}$ به صورت
$\frac{2a}{a^2 + \omega^2}$
است.
اثبات:

$$
\begin{gathered}
	x(t)=e^{-d t \mid}=\left\{\begin{array}{ll}
		e^{-a t} & t>0 \\
		e^{a t} & t<0
	\end{array}\right. \\
	X(\omega)=\int_{-\infty}^{0} e^{a t} e^{-j \omega t} d t+\int_{0}^{\infty} e^{-a t} e^{-j \omega t} d t \\
	=\int_{-\infty}^{0} e^{(a-j \omega) t} d t+\int_{0}^{\infty} e^{-(a+j \omega) t} d t \\
	=\frac{1}{a-j \omega}+\frac{1}{a+j \omega}=\frac{2 a}{a^{2}+\omega^{2}}
\end{gathered}
$$

همچنین می دانیم که تبدیل فوریه
$t f(t)$
تبدیل فوریه برابر است با:
$j \frac{d}{d\omega} F(\omega)$

پس در این جا هم جواب

$$j \frac{d}{d \omega } \frac{2a}{a^2 + \omega^2} = -\frac{4 a  j \omega}{\left(a^2+\omega^2\right)^2}$$


\subsubsection{بخش d}

$$X(jw) = \int_{-\infty}^{\infty} \cos (\omega_0 t) u(t) e^{-j \omega t} dt$$

$$X(jw) = \int_{0}^{\infty} \cos (\omega_0 t) e^{-j \omega t} dt$$

$$X(jw) = \frac{1}{2} \int_{0}^{\infty} (e^{j \omega_0 t} + e^{-j \omega_0 t}) e^{-j \omega t} dt$$
$$= -\frac{j \omega}{\omega^2-{\omega_0}^2}$$


\subsubsection{بخش e}

$$
\Delta(t)=\left\{\begin{array}{lr}
	1-2|t| & 0 \leq t \leq 1 / 2 \\
	0 & \text { otherwise }
\end{array}\right. = 
\left\{\begin{array}{lr}
	1-2t & 0 \leq t \leq 1 / 2 \\
	0 & \text { otherwise }
\end{array}\right.
$$


$$
\begin{aligned}
	F(\omega) &=\int_{-\infty}^{\infty} \Delta(t) e^{-j \omega t} d t \\
	&=\int_{0}^{1 / 2}(1-2 t) e^{-j \omega t} d t \\
	&=\left.\frac{j \omega(2 t-1)+2}{(j \omega)^{2}} e^{-j \omega t}\right|_{t=0} ^{1 / 2} \\
	&=\frac{2-j \omega-2 e^{-j \omega / 2}}{\omega^{2}}
\end{aligned}
$$

\subsubsection{بخش f}

$$
x(t)=\left\{\begin{aligned}
	1 & \text { if } 1 \leq|t| \leq 3 \\
	-1 & \text { if }|t|<1 \\
	0 & \text { otherwise }
\end{aligned}\right.
$$

می‌دانیم که تبدیل فوریه سیگنال مستطیلی بین $-1/2$ تا $1/2$ به صورت:
$\frac{\sin \frac{\omega }{2}}{\frac{\omega}{2}} = sinc (\omega/2)$
است. اگر سیگنال مستطیلی ذکر شده را با نماد $\Pi(t)$ نمایش بدهیم، عبارت بالا
$\Pi(t/6) - 2\Pi (t/2)$
است. در نتیجه
$$F(\omega) = 6sinc(6 \omega/2) - 4 sinc(2 \omega /2) = 6sinc(3\omega) - 4 sinc(\omega)$$

\newpage

\subsection{سوال دوم}

\subsubsection{بخش a}

$$
F(\omega)=\frac{16-16 j \omega+4 \omega^{2}-4 j \omega^{3}}{54+81 j \omega+18 \omega^{2}+31 j \omega^{3}-6 \omega^{4}}
$$

$$F(\omega) = \frac{4 (-2 + j\omega) (-1 + j\omega) (2 + j\omega)}{-(3 + j\omega)^2 (-3 + 2 j\omega) (2 + 3 j\omega)}$$

$$= \frac{80}{63 (j\omega+3)^2}+\frac{28}{1053 (2 j\omega-3)}+\frac{640}{637 (3 j\omega+2)}-\frac{4028}{3969 (j\omega+3)}$$

$$\frac{80}{63} t e^{-3t} u(t) +\frac{-14}{1053}e^{\frac{3}{2}t}u(-t)+\frac{640}{1911}e^{\frac{-2}{3} t}u(t) + \frac{4028}{3969} e^{-3t}u(t) $$

\subsubsection{بخش b}

$$F(j\omega) = 2\pi j \omega e^{-|\omega|}$$


$$x(t) =\frac{1}{2\pi}\int_{-\infty}^{\infty} 2\pi j \omega e^{-|\omega|} e^{j \omega t} d \omega$$

$$ \int_{-\infty}^{0} j\omega e^{\omega} e^{j \omega t} d\omega + \int_{0}^{\infty} j\omega e^{-\omega} e^{j\omega t} d\omega$$

$$=\frac{i}{(t-i)^2} +(-\frac{i}{(t+i)^2}) $$

$$=-\frac{4 t}{\left(t^2+1\right)^2}$$

\newpage
\subsection{سوال سوم}

برای این سوال ابتدا تبدیل فوریه سه سیگنال داده شده را بدست می آوریم.

برای $sinc(t)$ می دانیم که تبدیل فوریه آن
$
X_2(j\omega)= \operatorname{rect}(\frac{\omega}{2\pi})
$
است و تابع $rect$ به صورت زیر تعریف می شود:

$$ 
\operatorname{rect}(t)=\Pi(t)=\left\{\begin{array}{ll}
	0, & |t|>\frac{1}{2} \\
	\frac{1}{2}, & |t|=\frac{1}{2} \\
	1, & |t|<\frac{1}{2}
\end{array}\right.
$$

یعنی

$$X_2 (j\omega) = \left\{\begin{array}{ll}
	0, & |\omega|>\pi \\
	\frac{1}{2}, & |\omega|=\pi \\
	1, & |\omega|<\pi
\end{array}\right.$$

همچنین برای 
$sinc^2(t)$
می دانیم که تبدیل فوریه آن به نوعی کانلوشون دو سیگنال مربعی است. در نتیجه تبدیل فوریه آن به صورت مثلثی 
$X_1 (j\omega) = X_3(j\omega) =\operatorname{tri}(\frac{\omega}{2\pi})$
است و سیگنال مثلثی به صورت زیر تعریف می‌شود:

$$\operatorname{tri}(x) = \Lambda(x)=
=\left\{\begin{array}{ll}
	1-|x|, & |x|<1 \\
	0 & \text { otherwise }
\end{array}\right.
$$


$$X_1 (j\omega) = X_3(j\omega) ==\left\{\begin{array}{ll}
	1-|\frac{\omega}{2\pi}|, & |\omega|<2\pi \\
	0 & \text { otherwise }
\end{array}\right. $$

اگر سیستم $LTI$ باشد، عملا باید
$\operatorname{tri}(\frac{\omega}{2\pi}) \times H_a(j\omega) =
\operatorname{rect}(\frac{\omega}{2\pi})$
باشد. در نتیجه باید فرض کنیم که $A$ عملا به این صورت تعریف بشود:

$$H_a(j\omega)=\left\{\begin{array}{ll}
 \frac{1}{X_1 (\omega)}, & |\omega|<\pi \\
	0 & \text { otherwise }
\end{array}\right. = \left\{\begin{array}{ll}
1-|\frac{\omega}{2\pi}|, & |\omega|<\pi \\
0 & \text { otherwise }
\end{array}\right. 
$$

چنین تابعی تابع معتبری است. اعدادی بین $-\pi$ تا $\pi$ را به عددی غیر صفر نظیر می‌کند و بقیه را هم به صفر نظیر می‌کند. تابع دیگری که در آن ضرب شده هم در بین $-2\pi$ تا $\pi$ مقادیر غیر صفر دارد. پس نتایج ما منطقی است و سیستم $a$ \textbf{می‌تواند} LTI باشد.

اما برای حالت دوم و سیستم $b$، باید توجه کرد که این سیستم باید به این صورت باشد که
$$X_2(j\omega )\times H_b(j\omega) = X_3(j\omega)$$ 
بشود. نکته‌ای که وجود دارد این است که $X_2(j\omega)$ در بازه‌های 
$|x-2\pi|<\pi$
یعنی
$x\in (-2\pi , -\pi) \cup (-\pi , 2\pi)$
برابر صفر است. این عبارت در $H_b(j\omega)$ ضرب شده است و در همین بازه خروجی ناصفر تولید کرده‌است. در نتیجه $H_b(j\omega)$ نمی‌تواند تابع باشد و رابطه متعارفی نیست. در نتیجه سیستم $b$ \textbf{نمی‌تواند} \lr{LTI} باشد.



\newpage

\subsection{سوال چهارم}

$$h(t)=\frac{\sin (10 \pi  t)-\sin (6 \pi  t)}{2 \pi  t}$$

$$H(j\omega) = 
\left\{\begin{array}{ll}
	\frac{1}{2}, & |\omega|<10\pi \\
	0, & |\omega|>10\pi
\end{array}\right.
 -
 \left\{\begin{array}{ll}
 	\frac{1}{2}, & |\omega|<6\pi \\
 	0, & |\omega|>6\pi
 \end{array}\right.
 $$
 
 
 $$x(t) = \frac{1}{2} (\cos (5 \pi  t)+\cos (9 \pi  t))$$
 
 
 $$X(j\omega) = \frac{\pi}{2} (\delta(\omega-5\pi) + \delta(\omega + 5\pi)) + \frac{\pi}{2} (\delta(\omega-9\pi) + \delta(\omega + 9\pi))$$
 
 با ضرب $H$ در $X$ ، عبارت داری $5\pi$ که فقط در $5\pi$ مقدار دارد، در هر دو حالت $H$ شامل حالت $\frac{1}{2}$ شده و صفر می‌شود. ولی عبارت دومی فقط در حالت $|w|<10\pi$ صدق می‌کند و نصف می‌شود. در نتیجه:
 
 $$Y(j\omega) =\frac{\pi}{4} (\delta(\omega-9\pi) + \delta(\omega + 9\pi))$$
 
 پس
 
 $$y(t) = \frac{1}{4} \cos (9 \pi t)$$
 
 
\subsection{سوال پنجم}

$$
2 \frac{d^{2} y(t)}{d t^{2}}+3 \frac{d y(t)}{d t}-2 y(t)=x(t-1)
$$

با فرض شرایط اولیه صفر:

$$2 (j\omega)^2 Y(j\omega) + 3 (j\omega) Y(j\omega) - 2 Y(j\omega) = e^{-j\omega} X(j \omega)$$

$$X(j \omega) = \frac{1}{5 + j\omega} - \frac{1}{1 +j \omega}$$


$$Y(j \omega) = e^{- j \omega} \times \frac{\frac{1}{j\omega+5}-\frac{1}{j\omega+1}}{2 j\omega^2+3 j\omega-2}$$

$$= e^{-j \omega}\times( -\frac{4}{(j\omega+1) (j\omega+5) \left(2 j\omega^2+3 j\omega-2\right)})$$
$$= e^{-j \omega} \times( -\frac{4}{15 (j\omega+2)}+\frac{1}{33 (j\omega+5)}-\frac{32}{165 (2 j\omega-1)}+\frac{1}{3 (j\omega+1)})$$

ابتدا قسمت درون پرانتز را تبدیل فوریه معکوس می گیریم:

$$\rightarrow \frac{-4}{15} e^{-2t}u(t) + \frac{1}{33} e^{-5t}u(t) + \frac{16}{65} e^{\frac{1}{2} t} u(-t) + \frac{1}{3} e^{-t}u(t) $$

حال اثر $e^{-j \omega}$ را که شییفت به راست می دهد را اعمال می‌کنیم:


$$y(t) = \frac{-4}{15} e^{-2(t-1)}u(t-1) + \frac{1}{33} e^{-5(t-1)}u(t-1) + \frac{16}{65} e^{\frac{1}{2} (t-1)} u(-(t-1)) + \frac{1}{3} e^{-(t-1)}u(t-1) $$
$$=\frac{-4}{15} e^{-2t+2)}u(t-1) + \frac{1}{33} e^{-5t+5)}u(t-1) + \frac{16}{65} e^{\frac{1}{2} t-\frac{1}{2}} u(-t+1) + \frac{1}{3} e^{-t+1}u(t-1)$$

\newpage
\subsection{سوال ششم}

$$
\sum_{k=-\infty}^{+\infty} \delta\left(\omega-k \omega_{0}\right)=\frac{1}{\omega_{0}} \sum_{n=-\infty}^{+\infty} e^{\frac{2 \pi n j \omega}{\omega_{0}}}
$$

اگر توجه کنیم عبارت صورت سوال به شدت شبیه سری فوریه است. در اصل عبارت
$\sum_{k=-\infty}^{+\infty} \delta\left(\omega-k \omega_{0}\right)$
یک عبارت متناوب با دوره تناوب $\omega_0$ است که در هر $\omega_0$ یک تابع ضربه ایجاد کرده است. در نتیجه ضرایب فوریه آن را در یک دوره تناوب بدست می‌آوریم. البته بهتر بود به جای نماد $\omega_0$ از نماد $T_0$ استفاده می‌شد چون عملا این جا $\omega_0$ فرکانس نیست و خود دوره تناوب است ولی به هر حال با همین نماد جلو می‌رویم. عملا بهتر بود برای رعایت نمادگذاری به جای $\omega$ هم $t$ گذاشته می‌شد ولی در صورت سوال نمادگذاری متفاوتی استفاده شده است و از آن جایی که عملا تبدیل خاصی هم خواسته نشده است، می‌توانیم به صورت سوال به چشم یک تابع معمولی نگاه کنیم که به جای نماد $t$ نماد $\omega$ در آن گذاشته شده است.

$$a_k = \frac{1}{\omega_0} \int_{-\omega_0/2}^{\omega_0/2} \sum_{n=-\infty}^{\infty} \delta (\omega - n \omega_{0} ) e^{-jk\frac{\omega_0}{2 \pi} \omega} d\omega$$

عبارت بالا فقط به ازای $n=0$ مقدار غیر صفر دارد (در بازه انتگرال نوشته شده):

$$a_k = \frac{1}{\omega_0} \int_{-\omega_0/2}^{\omega_0 /2} \delta(\omega) e^{-j k \frac{\omega_0}{2\pi}\omega}  d\omega$$

عبارت بالا تنها در $\omega=0$ ناصفر است پس:

$$a_k = \frac{1}{\omega_0} \int_{-\omega_{0}/2}^{\omega_0/2} \delta(\omega) d\omega = \frac{1}{\omega_0}$$

در نتیجه با توجه به رابطه سری فوریه به عبارت زیر می رسیم. توجه کنید که در این جا عملا
 $\omega_0$
  رابطه فوریه به صورت 
   $\frac{\omega_0}{2\pi}$
   و $t$ آن رابطه به صورت $\omega$ است.
   

$$\sum_{k=-\infty}^{+\infty} \delta\left(\omega-k \omega_{0}\right)= \sum_{n=-\infty}^{+\infty} a_n e^{\frac{2 \pi n j \omega}{\omega_{0}}}$$

$$\sum_{k=-\infty}^{+\infty} \delta\left(\omega-k \omega_{0}\right)=\frac{1}{\omega_{0}} \sum_{n=-\infty}^{+\infty} e^{\frac{2 \pi n j \omega}{\omega_{0}}}$$

\newpage
\subsection{سوال هفتم}

تعریف تکیه‌گاه یا \lr{Support}:  به شکل کلی مجموعه‌ای از نقاط که تابع به ازای آن‌ها صفر نباشد را \lr{Support} تابع می‌گویند.



قضیه را به شکل کلی اثبات می‌کنیم.

تابع را با $f$ و تبدیل فوریه آن را با $F$ نشان می‌دهیم.

در این سوال اثبات می‌کنیم که $f$ و  $F$ نمی‌توانند همزمان support متناهی داشته باشند مگر این که $f=0$ باشد. یعنی به جز تابع $0$ که تبدیل فوریه‌ اش هم $0$ است و عملا می‌توان گفت support ای ندارد، هیچ حالتی دیگری امکان ندارد هردوی آن‌ها همزمان متناهی باشند. البته در اصل اثباتی که این جا می‌نویسیم، برای حالت compact-support است ولی  عملا compact-support حالت finite-support را هم پوشش می‌دهد. compact-support نشان دهنده وجود یک بازه است که در آن مقدار تابع ناصفر است و پس از آن صفر است و عملا تعداد support متناهی را حالت خاصی از compact-support بدانیم.



فرض کنیم که $f$ پیوسته بوده و در بازه
$[-\pi/2 , pi/2]$
تعریف شده باشد. همچنین 
$F(\omega)$
به ازای 
$|\omega|>N$
برابر صفر باشد. نشان می‌دهیم که چنین حالتی تنها در صورتی که $f$ صفر باشد امکان پذیر است. برای این کار، $f$ را به صورت متناوب در نظر گرفته و دوره تناوب آن را بین
$[-\pi , \pi]$
قرار می‌دهیم. در این صورت ضرایب سری فوریه آن به صورت زیر می‌شود:

$$c_n = \frac{1}{2\pi} (\int_{-\pi}^{\pi} f(x) e^{j n x} dx)$$
عبارت داخل پرانتز عملا خود تبدیل فوریه $f$ است. یعنی
$c_n = \frac{1}{2\pi} F(n)]$
شده است. حال اگر به ازای
$|n|>N$
مقادیر تبدیل فوریه صفر باشند، یعنی عبارت بالا هم تنها به ازای تعداد محدوی عدد مقدار ناصفر دارد.

در نتیجه یعنی سری فوریه $f$ در بازه 
$[-\pi , \pi]$
یک جمع متناهی به صورت
$$f(x)= \sum_{n=-N}^{n=N} c_n e^{j n x}$$
است. این عبارت عملا یک چندجمله ای مثلثاتی از درجه $N$ (یا کمتر از $N$) است.

البته توجه کنید که ممکن است ابهاماتی پیرامون همگرایی پیش بیاید ولی از آن جایی که 
$\sum_{n=-\infty}^{\infty} |c_n| <\infty$
است (به دلیل متناهی بودن) می توانیم از همگرایی مطمئن باشیم.

حال نشان می‌دهیم که یک تابع چند جمله ای مثلثاتی که بعد از یک بازه‌ای کاملا صفر می‌شود باید متحد با صفر باشد.


$$
P_{N}(x)=\sum_{-N}^{N} c_{n} e^{i n x}=
$$


$$
\left(\sum_{-N}^{N} \alpha_{n} \cos n x+\beta_{n} \sin n x\right)+i\left(\sum_{-N}^{N} A_{n} \cos n x+B_{n} \sin n x\right)
$$
$$=u(x) + i v(x)$$

از طرفی می‌دانیم که توابع مثلثاتی بسط تیلور همگرا دارند. در نتیجه اگر مقدار آن حول نقطه ای خاص $0$ باشد، همه ضرایب تیلور آن باید صفر باشند. در نتیجه با توجه به این که در مثلا بازه
 $[\pi/2 , \pi]$
 مقدار تابع صفر است و با توجه به همگرایی بسط تیلور تابع، مقدار آن باید به ازای همه نقاط صفر بوده باشد. یعنی $f\equiv0$ بوده است.
 
 در نتیجه از متناهی بودن support هر دوی $f$ و $F$ نتیجه گرفتیم که $f$ صفر است. در نتیجه امکان ندارد هر دوی آن‌ها متناهی باشند.
 
 توجه کنید که بازه انتخاب شده برای این سوال اختیاری بود و می‌شد بازه‌های دیگری را هم انتخاب کرد و به راحتی با Scale کردن مقادیر، همچنان توضیحات بالا برقرار بود.
 
 
 البته تقریبا بدیهی بود که یک چندجمله ای مثلثاتی درجه $N$ حداکثر $2N$ ریشه دارد (این را هم می‌شود به راحتی با در نظر گرفتن صفحه مختلط و نوشتن توابع مثلثاتی به صورت مختلط اثبات کرد) و در نتیجه این که در بازه
 $[\pi/2 , \pi]$
 عبارت تماما صفر بود و بسط مثلثاتی متناهی از آن داشتیم،‌ نشان دهنده این بود که این عبارت باید متحد با صفر باشد. توجیه بسط تیلور صرفا برای کامل تر شدن اثبات بود.
 
 
 توجه کنید که در صورت سوال \lr{finite} بودن صحبت شده که می توان آن را مشابه \lr{compact} بودن در نظر گرفت ولی با بازه گسترده تر. چون \lr{compact} بودن و \lr{support} هم بر این اساس است که از یک بازه ای به بعد، همه مقادیر صفر بشوند و قبل از لزوما صفر نباشند. در حالت \lr{compact} می‌تواند تعداد این مقادیر بیشمار هم باشد و مثلا یک بازه پیوسته باشد ولی می‌تواند محدود هم باشد و مشکل خاصی از این بابت نیست.
 
 
 اثبات گفته شده در این جا تا حدی اثبات شهودی بود برای اثبات Rigorous ریاضی، بخش هایی از اثبات قضیه موسوم به Amrein-Berthier را می‌آوریم. اثبات کامل قضیه با همه ریزه‌کاری‌ها، فراتر از دانش مطرح شده در این درس است.
 
 برای اثبات ابتدا به قضیه Paley-Weiner توجه می‌کنیم. طبق این قضیه اگر $f$ در فضای لبگ 
 $L^2(\mathbb{R}^d)$
 با توپی به شعاع R ساپورت بشود، آن‌گاه تبدیل فوریه مختلط آن یک تابع همگانی (در همه نقاط ناصفر) است که شرط زیر را ارضا می‌کند:
 $$
 |\hat{f}(z)| \leqslant C e^{2 \pi R|z|}
 $$
 
 از روی این قضیه، اثبات می‌شود که اگر هر دوی $f$ و $\hat{f}$ ساپورت Compact داشته باشند، $f=0$ است. زیرا اگر $\hat{f}$ به صورت Compact ساپورت باشد، آنگاه طبق قضیه بالا $f$ محدود سازی یک تابع تحلیلی به $\mathbb{R}$ است. در نتیجه اگر $f$ به صورت Compact ساپورت باشد، همچنان صفر‌های آن ایزوله نیستند و در همه نقاط قرار دارند. پس تحلیلی بودن $f$ نتیجه می‌دهد که $f=0$
 
 خود قضیه Amrein-Berthier، می‌گوید که اگر
 $f \in L^2(\mathbb{R}^d), E,F \subset \mathbb{R}$
 بوده و $E,F$ دو Measure متناهی (Finite) باشند، آن‌گاه
 $$
 \|f\|_{L^{2}\left(\mathbb{R}^{d}\right)} \leqslant C\left(\|f\|_{L^{2}\left(E^{c}\right)}\|\hat{f}\|_{L^{2}\left(F^{c}\right)}\right)
 $$
 
 که این دقیقا شکل ریاضیاتی حکمی است که سوال قصد اثبات آن را دارد. در عبارت بالا $C$ فقط وابسته به $E,F,d$ است.
 
 برای اثبات این قضیه از لم دیگری استفاده می‌شود. این لم به این شکل است. فرض کنید 
 $E,F \subset \mathbb{R^d}$
 دو Measure متناهی باشند، آن‌گاه
 اگر وجود داشته باشد $C'$ به طوری که
 $
 \operatorname{supp}(f) \subset F \Longrightarrow\|f\|_{2} \leqslant C^{\prime}\|f\|_{L^{2}\left(E^{c}\right)}
 $
 
 آن‌؛اه وجود دارد ثابت $C$ که به ازای هر $f \in L^2(\mathbb{R}^d)$ داریم:
 $$
 \mid f \|_{2} \leqslant C\left(\|f\|_{L^{2}\left(E^{c}\right)}+\|\hat{f}\|_{L^{2}\left(F^{c}\right)}\right.
 $$
 
 از اثبات این لم صرف نظر می‌کنیم. با فرض درستی این لم، برای اثبات خود قضیه $E,F$ را فیکس کرده و عملگر $T$ را به صورت زیر تعریف می‌کنیم:
 
\makeatletter
\DeclareRobustCommand\widecheck[1]{{\mathpalette\@widecheck{#1}}}
\def\@widecheck#1#2{%
	\setbox\z@\hbox{\m@th$#1#2$}%
	\setbox\tw@\hbox{\m@th$#1%
		\widehat{%
			\vrule\@width\z@\@height\ht\z@
			\vrule\@height\z@\@width\wd\z@}$}%
	\dp\tw@-\ht\z@
	\@tempdima\ht\z@ \advance\@tempdima2\ht\tw@ \divide\@tempdima\thr@@
	\setbox\tw@\hbox{%
		\raise\@tempdima\hbox{\scalebox{1}[-1]{\lower\@tempdima\box
				\tw@}}}%
	{\ooalign{\box\tw@ \cr \box\z@}}}
\makeatother
 $$
 T f=\chi_{E}\widecheck{\left(\chi_{F} \hat{f}\right)}
 $$
 
 اگر $L^2$ نرم عملگر $T$ کمتر از $1$ باشد، فرض لم برقرار خواهد بود با
$$
C^{\prime}=\frac{1}{1-\|T\|}
$$
در نتیجه کافیست نشان بدهیم که 
$\|T\| <1$.

برای این موضوع باید توجه کرد که $T$ یک انتگرال Hilbert-Schmidt با کرنل
$$
K(x, y)=\chi_{E}(x) \hat{\chi}_{F}(y)
$$
بوده و نرم $\sigma$ برای انتگرال‌های Hilbert-Schmidt به صورت زیر است:

$$
\sigma^{2}=\|K\|_{L^{2}\left(\mathbb{R}^{2 d}\right)}^{2}=\int_{\mathbb{R}^{2 d}} \chi_{E}^{2}(x) \chi_{F}^{2}(y) d x d y=|E \| F|
$$

از این نتیجه می‌شود که $T$ یک عملگر Compact است. با توجه به Compact بودن، نرم $L^2$ عملگر $T$ یک خواهد بود اگر و تنها اگر وجود داشته باشد تابع
$f \in L^2(\mathbb{R}^d)$
به طوری که $f$ تحت ساپورت فضای $E$ و $\hat{f}$ تحت ساپورت $F$ باشد. بنابراین عبارت گفته شده در صورت قضیه معادل این است که یک تابع غیرصفر و تبدیل فوریه اش نمی‌توانند هر دو Support ای در یک فضای Finite داشته باشند.

حال می‌دانیم به این که نرم $T$ کمتر از $1$ است یعنی با توجه به این که $T$ حاصلضرب دو Projection و عملگر افکنش است، می‌دانیم نرم $L^2$ آن کمتر از ۱ است. با برهان خلف فرض کنید که 
$||T|| =1$
در این صورت
وجود دارد
$f \in L^2(\mathbb{R}^2)$
که
$\operatorname{supp}(f) \subset E , \operatorname{supp}(\hat{f}) \subset F  $

با تبدیل متناوب $f$ با مقادیر کوچکشونده $2^{-k}$ به مجموعه ای از بی‌نهایت تابع متسقل خطی می‌رسیم که روی مجموعه $E'$ به صورت Compact Support هستند و تبدیل فوریه همه آنان Support ای در $F$ دارد. از این جا نتیجه می‌شود که این توابع، توابع ویژه عملگر $T'$ هستند که با جایگزین کردن $E'$ به جای $E$ در تعریف $T$ و با مقدار ویژه $1$ بدست می‌ایند. اما از آن جایی که $T'$ یک عملگر Compact است، همانند T، فضای ویژه (eigenspace) آن که مقدار ویژه‌های ناصفر دارد، همگی متنهای-بعدی (finite-dimensional) هستند و به تناقض رسیدیم.

پس قضیه Amrein-Berthier اثبات شد.


اثبات بالا، در اصل اثبات خیلی ریاضیاتی حکم گفته شده در سوال بود. این اثبات برداشته شده از مقاله
\lr{THE UNCERTAINTY PRINCIPLE IN HARMONIC ANALYSIS}
نوشته
\lr{BLAINE TALBUT}
بود و لینک دانلود آن در زیر آورده شده است:
 \href{http://www.math.uchicago.edu/~may/REU2014/REUPapers/Talbut.pdf}{لینک دانلود}
 
 و البته همان طور که مشخص است، از مفاهیم ریاضیاتی بسیار پیشرفته‌تری نسبت به این درس در آن استفاده شده است و من سعی کردم در حد فهم خودم از آن‌ها و به شکل تا حدی مختصرتر و با کمی سرچ در مورد فهم نسبی مفاهیم استفاده شده، آن را در این جا بیاورم.
 
 

 \newpage
 
 \subsection{سوال هشتم}
 $$
 \begin{aligned}
 	&x_{1}(t)=\left\{\begin{array}{ll}
 		0 & t<0 \\
 		1 & t \geq 0
 	\end{array}\right. \\
 	&x_{2}(t)=\left\{\begin{array}{ll}
 		0 & t \leq 0 \\
 		1 & t>0
 	\end{array}\right.
 \end{aligned}
 $$
 
 
 \subsubsection{بخش a}
 
 برای تبدیل فوریه
 $$
 	x_{1}(t)=\left\{\begin{array}{ll}
 		0 & t<0 \\
 		1 & t \geq 0
 	\end{array}\right.
 $$
 از تبدیل فوریه 
 $$
 g_{\alpha}(t)=\left\{\begin{array}{ll}
 	e^{-a t} & t \geq 0 \\
 	0 & t<0
 \end{array}\right.
 $$
استفاده می‌کنیم.
برای این عبارت به ازای $a>0$ داریم:

$$H_{\alpha}(j\omega) =
\frac{1}{a+j \omega}=\frac{a-j \omega}{a^{2}+\omega^{2}}=\frac{a}{a^{2}+\omega^{2}}-\frac{j \omega}{a^{2}+\omega^{2}}
$$

در حد $a \to 0$ داریم:

$$\frac{a}{a^2 + \omega^2} \to \pi \delta(\omega)$$
و
$$-\frac{j \omega}{a^2 +\omega^2} \to \frac{1}{j\omega}$$

در نتیجه

$$\mathbb{F}(x_1(t)) = \pi \delta(\omega) + \frac{1}{j\omega}$$

برای تابع دوم هم عملا به همین شکل می‌شود. چون اساسا تبدیل فوریه تابع نمایی داده شده تغییر خاصی نمی‌کند و در نتیجه حد‌های آن هم به همین شکل خواهند ماند.


\subsubsection{بخش b و c}

نکته اساسی که در مورد این دو سیگنال وجود دارد این است که به هر حال در هر دو، در حوالی $t=0$ شاهد یک جهش ناگهانی در مقدار سیگنال هستیم. در یکی از مقدار صفر در $0^{-}$ به $1$ در خود $0$ جهش می‌کند و در دیگری از مقدار صفر در خود صفر به مقدار $1$ در $0^+$ جهش صورت می‌گیرد.

در اصل اگر بخواهیم خیلی ساده‌انگارانه بخواهیم با سوال رفتار کنیم، جواب قسمت قبل را به شکل ساده ای
$\frac{1}{j\omega}$
بدست می‌آوردیم. حالا اگر می‌خواستیم تبدیل معکوس آن را حساب کنیم داشتیم:


$$x(t) = \frac{1}{2\pi} \int_{-\infty}^{\infty} \frac{e^{j\omega t}}{j\omega} d\omega = \frac{1}{2 \pi j} (\int_{-\infty}^{0} ... + \int_{0}^{\infty} ...)$$
$$
=\frac{1}{2 \pi } \int_{0}^{\infty} \frac{e^{i \omega t}-e^{-i \omega t}}{\omega} d \omega=\frac{1}{\pi} \int_{0}^{\infty} \frac{\sin \omega t}{\omega} d \omega=\frac{1}{2}
$$

نکاتی در مورد این انتگرال وجود دارد. اول این که تقسیم آن به دو انتگرال نکته جالبی را اشکار می‌کند. این دو انتگرال جداگانه، دو قسمت حقیقی را تولید می‌کنند ولی با  ترکیب آن‌ها قسمت‌های حقیق ی حذف شده و قسمت موهومی می‌ماند که با تقسیم شدن بر یکه موهومی، حاصلی حقیقی به دست می‌دهد. تقسیم کردن انتگرال به دو انتگرال در نقطه صفر را اصطلاحا محاسبه \lr{Cauchy Principle Value} می‌گویند. نکته دیگر این است که نتیجه ما تنها در صورتی درست است که $t>0$ باشد. اگر $t<0$ باشد، انتگرال یک منفی دیگر‌هم تولید می‌کند و جواب $-1/2$ می‌شود. نکته این جاست که می‌بینیم اگر به شکل ساده انگارانه تبدیل فوریه $H(t)$ را
$\frac{1}{j\omega}$
بگیریم به عبارت
$\left\{\begin{array}{ll}
	\frac{1}{2} & t > 0 \\
	\frac{-1}{2} & t<0
\end{array}\right.$
می رسیم. یعنی عملا به $H(t)-1/2$ رسیده‌ایم که منظور از $H(t)$ همان تابع هوی ساید است.

نکته دیگری که وجود دارد این است که عملا خود نقطه صفر در این جا نقش خاصی ایفا نمی‌کند. در نتیجه چه نقطه صفر را تعریف شده و جزو بازه مربوط به اعداد مثبت و چه اعداد منفی و چه حتی تعریف نشده در نظر بگیریم در حاصل کار تفاوتی ایجاد نمی شود.

عبارت بالا نشان می‌دهد که مستقیما با فرمول ساده تبدیل فوریه نمی‌توانیم این عبارت را حل کنیم. ولی عبارتی که در بخش a قبل بدست آوردیم،‌ واقعا جواب درستی است.

برای درستی آن به این نکته توجه کنید که
$$\mathbb{F}^{-1} (\frac{1}{j\omega}) = \left\{\begin{array}{ll}
	\frac{1}{2} & t > 0 \\
	\frac{-1}{2} & t<0
\end{array}\right.$$

و

$$\mathbb{F}^{-1}{\pi \delta(\omega) } = \frac{1}{2}$$

در نتیجه

$$\mathbb{F}^{-1}{\pi \delta(\omega) + \frac{1}{j\omega}} = H(t)$$
که منظور در این جا $H$ تابع هوی ساید با هر تعریفی برای $t=0$ است.

در نتیجه مشاهده می‌کنیم که Duality بین این دو واقعا وجود دارد.


\subsection{بخش d}


در بالا دیدیم که عبارتی که داده شده بود، عملا یک به یک بودن تبدیل فوریه را بر هم می‌زد و توابع مختلفی به یک تبدیل فوریه می‌رسید. ادعا می‌کنیم تبدیل فوریه در صورتی Bijective است که \lr{Square-Integrable} باشد. یعنی در فضای
 لبگ
  $L^2(\mathbb{R}) \cup L^1(\mathbb{R})$
   باشد.
 .

 

برای اثبات bijective بودن باید surjective بودن و Injective بودن نشان داده شود. برای Injective بودن به طور کلی از قضیه‌ای موسوم به قضیه Plancherel استفاده می‌شود که تا حدی شبیه روابطی است که برای پارسوال داشتیم.


$$
\int_{-\infty}^{\infty}|f(x)|^{2} d x=\int_{-\infty}^{\infty}|\widehat{f}(\xi)|^{2} d \xi
$$

به طور دقیق تر، این رابطه می‌گوید که اگر یک تابع در فضای لبگ $L^1(\mathbb{R})$ و $L^2(\mathbb{R})$ باشد، تبدیل فوریه آن در فضای 
$L^2(\mathbb{R})$
بوده و تبدیل فوریه یک نگاشت ایزومتری نسبت به نرم $L^2$ است. به بیان دیگر یعنی اگر تبدیل فوریه محدود به
 $L^2(\mathbb{R}) \cup L^1(\mathbb{R})$
 باشد، می‌تواند به شکل یکتا به نگاشت ایزومتریک 
 $
 L^{2}(\mathbb{R}) \mapsto L^{2}(\mathbb{R})
 $
 گسترش می‌یابد که به آن تبدیل Plancherel می‌گویند. این تبدیل به طور کلی برای سایر فضاهای $n$ بعدی اقلیدسی $\mathbb{R}^N$ هم برقرا است.
 

برای اثبات یعنی اگر $E(t)$ را خود تابع و $E_v$ را تبدیل فوریه آن در نظر بگیریم داریم:


 $$
 \begin{aligned}
 	\int_{-\infty}^{\infty}|E(t)|^{2} d t &=\int_{-\infty}^{\infty} E(t) \bar{E}(t) d t \\
 	&=\int_{-\infty}^{\infty}\left[\int_{-\infty}^{\infty} E_{v} e^{-2 \pi i v t} d v \int_{-\infty}^{\infty} \bar{E}_{v^{\prime}} e^{2 \pi i V^{\prime} t} d v^{\prime}\right] d t \\
 	&=\int_{-\infty}^{\infty} \int_{-\infty}^{\infty} \int_{-\infty}^{\infty} E_{v} \bar{E}_{v^{\prime}} e^{2 \pi i t\left(v^{\prime}-v\right)} d v d v^{\prime} d t \\
 	&=\int_{-\infty}^{\infty} \int_{-\infty}^{\infty} \int_{-\infty}^{\infty} E_{v} \bar{E}_{v^{\prime}} e^{2 \pi i t\left(v^{\prime}-v\right)} d t d v d v^{\prime} \\
 	&=\int_{-\infty}^{\infty} \int_{-\infty}^{\infty} \delta\left(v^{\prime}-v\right) E_{v} \bar{E}_{v^{\prime}} d v d v^{\prime} \\
 	&=\int_{-\infty}^{\infty} E_{v} \bar{E}_{v} d v \\
 	&=\int_{-\infty}^{\infty}\left|E_{v}\right|^{2} d v
 \end{aligned}
 $$
 
 دلیل استفاده از این نماد این بود که در خود اصل قضیه Plancherel از این نوع نمادگذاری استفاده شده است.
 
 در اصل این قضیه منجر به Injectivity تبدیف لوریه می‌شود. به  بیان دیگر چون
 $$||f_1 -f_2||_2 = ||\hat{f}_1 - \hat{f_2}||_2$$
 داریم که تبدیل فوریه دو تابع $L^2$ با هم برابرند اگر و فقط اگر دو تابع اصلی با هم برابر باشند. منظور از $||x||_2$
 هم نرم-۲ تابع است.
 
 
 از سوی دیگر قضیه پارسوال که پیش تر هم آن را اثبات کردیم، Surjective بودن را اثبات می‌کند.
 
 به بیان دیگر اگر دو تابع مختلط $A$ و $B$ سری‌های فوریه زیر را داشته باشند:
 
 $$
 \begin{aligned}
 	&A(x)=\sum_{n=-\infty}^{\infty} a_{n} e^{i n x}\\
 	&B(x)=\sum_{n=-\infty}^{\infty} b_{n} e^{i n x}
 \end{aligned}
 $$
 
 داریم:
 
 $$
 \sum_{n=-\infty}^{\infty} a_{n} \overline{b_{n}}=\frac{1}{2 \pi} \int_{-\pi}^{\pi} A(x) \overline{B(x)} \mathrm{d} x
 $$
 
 $$
 \begin{aligned}
 	\frac{1}{2 \pi} \int_{-\pi}^{\pi} A(x) \overline{B(x)} \mathrm{d} x&
 	&=\frac{1}{2 \pi} \int_{-\pi}^{\pi}\left(\sum_{n=-\infty}^{\infty} a_{n} e^{i n x}\right)\left(\sum_{n=-\infty}^{\infty} \overline{b_{n}} e^{-i n x}\right) \mathrm{d} x \\
 	&=\frac{1}{2 \pi} \int_{-\pi}^{\pi}\left(a_{1} e^{i 1 x}+a_{2} e^{i 2 x}+\cdots\right)\left(\overline{b_{1}} e^{-i 1 x}+\overline{b_{2}} e^{-i 2 x}+\cdots\right) \mathrm{d} x \\
 	&=\frac{1}{2 \pi} \int_{-\pi}^{\pi}\left(a_{1} e^{i 1 x} \overline{b_{1}} e^{-i 1 x}+a_{1} e^{i 1 x} \overline{b_{2}} e^{-i 2 x}+a_{2} e^{i 2 x} \overline{b_{1}} e^{-i 1 x}+a_{2} e^{i 2 x} \overline{b_{2}} e^{-i 2 x}+\cdots\right) \mathrm{d} x \\
 	&=\frac{1}{2 \pi} \int_{-\pi}^{\pi}\left(a_{1} \overline{b_{1}}+a_{1} \overline{b_{2}} e^{-i x}+a_{2} \overline{b_{1}} e^{i x}+a_{2} \overline{b_{2}}+\cdots\right) \mathrm{d} x
 \end{aligned}
 $$
 
 $$
 \begin{aligned}
 	&=\frac{1}{2 \pi}\left[a_{1} \overline{b_{1}} x+i a_{1} \overline{b_{2}} e^{-i x}-i a_{2} \overline{b_{1}} e^{i x}+a_{2} \overline{b_{2}} x+\cdots\right]_{-\pi}^{+\pi} \\
 	&=\frac{1}{2 \pi}\left(2 \pi a_{1} \overline{b_{1}}+0+0+2 \pi a_{2} \overline{b_{2}}+\cdots\right) \\
 	&=a_{1} \overline{b_{1}}+a_{2} \overline{b_{2}}+\cdots
 \end{aligned}
 $$
 $$=\sum_{n=-\infty}^{\infty} a_{n} \overline{b_{n}}$$
 
 در اصل قضیه پارسوال Surjective بودن (پوشا بودن) تبدیل فوریه را نشان می‌دهد. در اصل این موضوع براساس قضایای فضای هیلبرت توجیه می‌شود.
 
 با توجه به ایزومتری بودن تبدیل فوریه $\mathbb{F}$ در یک فضای بسته
 $L^2{R^n}$
 برای اثبات پوشا بودن از فرض خلف استفاده می‌کنیم. یعنی فرض می‌کنیم فضایی که این تبدیل نشان می‌دهد، همه
 $L^2(\mathbb(R)^n)$
 نباشد. در آن صورت تابع $g$ داریم که 
 $
 \int_{\mathbb{R}^{n}} \hat{f} g d x=0 \text { for all } f \in L^{2}\left(\mathbb{R}^{n}\right)
 $
 و
 $||g||\neq 0$
 با توجه به این که قضیه پارسوال و Plancherel برقرا هستند این موضوع برای همه $L^2$ صادق است یعنی داریم:
 $$
 \int_{\mathbb{R}^{n}} f \hat{g} d x=\int_{\mathbb{R}^{n}} \hat{f} g d x=0 .
 $$
 
 و یعنی
 $\hat{g}=0$
 که با فرض $
 \|g\|=\|\hat{g}\| \neq 0
 $
 در تناقض است. پس $\mathbb{F}$ باید همه فضای
 $L^2(\mathbb{R}^n))
$
را پوشش داده باشد. در اصل در اثبات بالا در ابتدا هنگام تعریف $g$ از این قضیه در فضاهای هیلبرت استفاده شده است که اگر $H$ یک فضای هیلبرت باشد و $K$ یک زیرفضای بسته از این فضا باشد که $K\neq H$ آن‌گاه مکمل عمود $K$ یعنی$
K^{\perp}
$
غیربدیهی است (بدیهی بودن یعنی در همه جا صفر بودن، غیربدیهی یعنی حداقل در بعضی نقاط غیر صفر است).

 
 توجه کنید که در اثبات‌های بالا، بعضا از تعریف ریاضیاتی تبدیل فوریه استفاده کرده‌ایم که در یک ضریب ثابت با تبدیل گفته شده در کتاب اپنهایم فرق دارد ولی در کل فرقی در اصل قضیه ایجاد نمی‌شود.
 
 
 
 منابع استفاده شده:
 
  \href{https://en.wikipedia.org/wiki/Plancherel_theorem}{منبع 1}

 \href{https://en.wikipedia.org/wiki/Parseval\%27s_identity}{منبع 2}
 
  \href{https://en.wikipedia.org/wiki/Parseval\%27s_theorem}{منبع 3}

 \href{https://math.stackexchange.com/questions/2024619/fourier-transform-is-a-bijective-transformation}{منبع 4}

 \href{https://mathworld.wolfram.com/PlancherelsTheorem.html}{منبع 5}
 
  \href{https://mathworld.wolfram.com/ParsevalsTheorem.html}{منبع 6}
  
   \href{https://web.stanford.edu/class/ee102/lectures/fourtran}{منبع 7}
   
    \href{https://www.cs.uaf.edu/~bueler/M611heaviside.pdf}{منبع 8}
 \newpage
\section{سوال عملی}

\subsection{بخش ۱}

نشان می‌دهیم که تبدیل معکوس فوریه گسسته به صورت زیر است:

$$x[n] = \frac{1}{N} \sum_{k=0}^{N-1} X_k e^{2\pi j k n / N}$$

که در آن $X_k$ خود تبدیل فوریه گسسته است.

برای اثبات داریم:

$$\frac{1}{N} \sum_{k=0}^{N-1} X_k e^{2\pi j k n /N} = \frac{1}{N} \sum_{k=0}^{N-1}(\sum_{m=0}^{N-1} x_m e^{- 2\pi j k m/N})e^{2 \pi j k n /N}$$

$$= \frac{1}{N}\sum_{m=0}^{N-1}\sum_{k=0}^{N-1} x_m e^{2\pi j k(n-m)/N} = \sum_{m=0}^{N-1} x_m (\frac{1}{N} \sum_{k=0}^{N-1} e^{2 \pi j (n-m)/N})$$
$$=\sum_{m=0}^{N-1} x_m \delta[n-m] = x_n$$

 
\end{document}



