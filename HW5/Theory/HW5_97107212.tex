\documentclass[12pt]{article}
\usepackage{graphicx,import}
\usepackage[svgnames]{xcolor} 
\usepackage{fancyhdr}
\usepackage{subfig}
\usepackage{hyperref}
\usepackage{enumitem}
\usepackage{cite}
\usepackage{fancyvrb}
\usepackage[many]{tcolorbox}
\usepackage{listings }
\usepackage[a4paper, total={6in, 8in} , bottom = 25mm , top = 25mm, headheight = 1.25cm , includehead,includefoot,heightrounded ]{geometry}
\usepackage{afterpage}
\usepackage{amssymb}
\usepackage{pdflscape}
\usepackage{gensymb}
\usepackage{textcomp}
\usepackage{tikz,pgfplots}
\usepackage{xecolor}
\usepackage{rotating}
\usepackage{pdfpages}
\usepackage[Kashida]{xepersian}
\usepackage[T1]{fontenc}
\usepackage{tikz}
\usepackage[utf8]{inputenc}
\usepackage{PTSerif} 
\usepackage{seqsplit}

\usepackage[edges]{forest}

\usepackage{listings}
\usepackage{xcolor}

\hypersetup{
	colorlinks   = true, %Colours links instead of ugly boxes
	urlcolor     = blue, %Colour for external hyperlinks
	linkcolor    = blue, %Colour of internal links
	citecolor   = red %Colour of citations
}
 
\definecolor{codegreen}{rgb}{0,0.6,0}
\definecolor{codegray}{rgb}{0.5,0.5,0.5}
\definecolor{codepurple}{rgb}{0.58,0,0.82}
\definecolor{backcolour}{rgb}{0.95,0.95,0.92}
 
\NewDocumentCommand{\codeword}{v}{
\texttt{\textcolor{blue}{#1}}
}
\lstset{language=java,keywordstyle={\bfseries \color{blue}}}

\lstdefinestyle{mystyle}{
    backgroundcolor=\color{backcolour},   
    commentstyle=\color{codegreen},
    keywordstyle=\color{magenta},
    numberstyle=\tiny\color{codegray},
    stringstyle=\color{codepurple},
    basicstyle=\ttfamily\normalsize,
    breakatwhitespace=false,         
    breaklines=true,                 
    captionpos=b,                    
    keepspaces=true,                 
    numbers=left,                    
    numbersep=5pt,                  
    showspaces=false,                
    showstringspaces=false,
    showtabs=false,                  
    tabsize=2
}

\lstset{style=mystyle}

\settextfont[Scale=1.2 ,BoldFont={Bahij Nazanin-Bold.ttf} , ItalicFont = {IRNazaninIranic.ttf}]{Bahij Nazanin-Regular.ttf}
\setlatintextfont[Scale = 1.0]{Garamond}
\DefaultMathsDigits 
\DeclareMathSizes{11}{19}{13}{9} 
%\DeclareMathSizes{12}{14.4}{8}{9}





\newenvironment{changemargin}[2]{%
\begin{list}{}{%
\setlength{\topsep}{0pt}%
\setlength{\leftmargin}{#1}%
\setlength{\rightmargin}{#2}%
\setlength{\listparindent}{\parindent}%
\setlength{\itemindent}{\parindent}%
\setlength{\parsep}{\parskip}%
}%
\item[]}{\end{list}}


\definecolor{foldercolor}{RGB}{124,166,198}

\tikzset{pics/folder/.style={code={%
    \node[inner sep=0pt, minimum size=#1](-foldericon){};
    \node[folder style, inner sep=0pt, minimum width=0.3*#1, minimum height=0.6*#1, above right, xshift=0.05*#1] at (-foldericon.west){};
    \node[folder style, inner sep=0pt, minimum size=#1] at (-foldericon.center){};}
    },
    pics/folder/.default={20pt},
    folder style/.style={draw=foldercolor!80!black,top color=foldercolor!40,bottom color=foldercolor}
}

\forestset{is file/.style={edge path'/.expanded={%
        ([xshift=\forestregister{folder indent}]!u.parent anchor) |- (.child anchor)},
        inner sep=1pt},
    this folder size/.style={edge path'/.expanded={%
        ([xshift=\forestregister{folder indent}]!u.parent anchor) |- (.child anchor) pic[solid]{folder=#1}}, inner xsep=0.6*#1},
    folder tree indent/.style={before computing xy={l=#1}},
    folder icons/.style={folder, this folder size=#1, folder tree indent=3*#1},
    folder icons/.default={12pt},
}

\begin{document}


%%% title pages
\begin{titlepage}
\begin{center}
        
\vspace*{0.7cm}

\includegraphics[width=0.4\textwidth]{sharif1.png}\\
\vspace{0.5cm}
\textbf{ \Huge{\emph ‌سیگنال‌ها و سیستم‌ها} }\\
\vspace{0.5cm}
\textbf{ \Large{ تمرین پنجم} }
\vspace{0.2cm}
       
 
      \large \textbf{دانشکده مهندسی کامپیوتر}\\\vspace{0.2cm}
    \large   دانشگاه صنعتی شریف\\\vspace{0.2cm}
       \large   ﻧﯿﻢ سال دوم 00-99 \\\vspace{0.2cm}
      \noindent\rule[1ex]{\linewidth}{1pt}
استاد:\\
    \textbf{{جناب آقای دکتر منظوری شلمانی}}


    \vspace{0.15cm}
نام و نام خانوادگی:\\

       
    \textbf{{امیرمهدی نامجو - 97107212}}
\end{center}
\end{titlepage}
%%% title pages


%%% header of pages
\newpage
\pagestyle{fancy}
\fancyhf{}
\fancyfoot{}
\cfoot{\thepage}
\chead{تمرین پنجم}
\rhead{\includegraphics[width=0.1\textwidth]{sharif.png}}
\lhead{امیرمهدی نامجو}
%%% header of pages

\KashidaOff

\section{سری فوریه}
\subsection{سوال اول}

\subsubsection{بخش a}


$$
f(x)=\left\{\begin{array}{lr}
	\pi-x & 0 \leq x \leq \pi \\
	x-\pi & -\pi \leq x \leq 0
\end{array}\right.
$$

ابتدا عامل DC را بدست می‌آوریم:

$$a_0 = \frac{1}{2\pi} \int f(x) dx = \frac{1}{2\pi} \int_{-\pi}^{0} (x- \pi) dx + \int_{0}^{\pi} (\pi -x) dx$$
$$= \frac{1}{2\pi} (\frac{-3\pi^2}{2} - \frac{\pi^2}{2}) = \boxed{- \pi} $$

برای ضرایب کسینوسی داریم:

$$a_n = \frac{1}{\pi} \int (f(x) \cos (nx)) dx$$

$$= \frac{1}{\pi} (\int_{-\pi}^{0} (x - \pi) \cos (nx) dx + \int_{0}^{\pi} (\pi - x) \cos (nx) dx)$$

ابتدا لازم است اشاره کنیم که
$$\int x \cos (nx) = \frac{1}{n^2} \cos(nx) + \frac{1}{n} x \sin (nx)$$

با توجه به این موضوع، از عبارت بالا می‌توان به راحتی انتگرال گرفت:


$$= \frac{1}{\pi}( (\frac{\cos (n x)}{n^2}+\frac{x \sin (n x)}{n}-\frac{\pi  \sin (n x)}{n}) |_{-\pi}^{0} + (-\frac{\cos (n x)}{n^2}+\frac{\pi  \sin (n x)}{n}-\frac{x \sin (n x)}{n})|_{0}^{\infty})$$

بعد از ساده سازی و محاسبات داریم:

$$\boxed{a_n = -\frac{2 (\pi  n \sin (\pi  n)+\cos (\pi  n)-1)}{\pi  n^2}}$$


برای محاسبات ضریب سینوسی داریم:

$$b_n = \frac{1}{\pi} \int (f(x) sin(nx)) dx$$

$$= \frac{1}{\pi} (\int_{-\pi}^{0} (x - \pi) \sin (nx) dx + \int_{0}^{\pi} (\pi - x) \sin (nx) dx)$$


ابتدا لازم است اشاره کنیم که
$$\int x \sin(nx) = \frac{\sin (n x)}{n^2}-\frac{x \cos (n x)}{n} $$

با توجه به این موضوع، عبارت بالا مانند بخش قبل به راحتی قابل محاسبه است. جوابی که در نهایت به آن می رسیم به صورت زیر است:


$$\boxed{b_n = \frac{2-2 \cos (\pi  n)}{n}}$$


و جواب نهایی به صورت:

$$f(x) = a_0 + \sum_{n=1}^{N} (a_n \cos (n x) + b_n \sin (n x))$$

خواهد بود. (تقسیم بر ۲ فرمول $a_0$ را به نوعی در خود انتگرال آن تاثیر داده‌ام)

کد آن در فایل 
\lr{\Verb+P1\_Q1\_a.py+}
قرار دارد. نمودار در صفحه بعد قرار گرفته است. شکل بالایی خود تابع و شکل های بعدی به ازای $N=2,5,20,50$ هستند.

\begin{center}
	\includegraphics[width = 1.0 \textwidth]{images/1.pdf}
\end{center}


\newpage

\subsubsection{بخش b}

$$
f(x)=\left\{\begin{array}{lr}
	1 & 0 \leq x<\frac{\pi}{2} \\
	0 & \frac{\pi}{2} \leq x<\pi \\
	0 & -\pi \leq x<0
\end{array}\right.
$$

$$a_0 = \frac{1}{2\pi} \int f(x) dx = \frac{1}{2\pi} \int_{0}^{\pi/2} 1 dx = \boxed{ \frac{1}{4}}$$


$$a_n = \frac{1}{\pi} \int (f(x) \cos (nx)) dx$$

$$=\frac{1}{\pi}\int_{0}^{\pi/2} \cos (nx) dx = \frac{1}{\pi }\frac{\sin(nx)}{n}|_{0}^{\pi/2}  = \boxed{\frac{\sin \left(\frac{\pi  n}{2}\right)}{\pi  n}}$$

$$b_n = \frac{1}{\pi} \int (f(x) \sin (nx)) dx$$
$$= \frac{1}{\pi} \int_{0}^{\pi/2} \sin (nx) dx = - \frac{1}{\pi} \frac{\cos (n x)}{n} |_{0}^{\pi/2}  = \boxed{\frac{2 \sin ^2\left(\frac{\pi  n}{4}\right)}{\pi  n}}$$

که در بالا از اتحاد
$\cos(2 \theta) = 1 - 2\sin^2 (\theta)$
استفاده شده است.


و جواب نهایی به صورت:

$$f(x) = a_0 + \sum_{n=1}^{N} (a_n \cos (n x) + b_n \sin (n x))$$



کد آن در فایل 
\lr{\Verb+P1\_Q1\_b.py+}
قرار دارد. نمودار در صفحه بعد قرار گرفته است. شکل بالایی خود تابع و شکل های بعدی به ازای $N=2,5,20,50$ هستند.

\begin{center}
	\includegraphics[width = 1.0 \textwidth]{images/2.pdf}
\end{center}

\newpage
\subsection{سوال دوم}
\subsubsection{بخش a}

$$\cos(4t) = \frac{1}{2} e^{-4 j t}+\frac{1}{2} e^{4 j t} $$

$$\sin(6t) = \frac{1}{2j} e^{6 j t}-\frac{1}{2j} e^{-6 j t} $$

در نتیجه ضرایب سری فوریه برای 
$\cos(4t)+\sin(6t)$
به صورت زیر است:

$$a_4 = \frac{1}{2} , a_{-4} = \frac{1}{2} , a_{6} = \frac{1}{2 j} , a_{-6} = \frac{-1}{2j}$$
و به ازای $k\neq \pm4,\pm6$ داریم $a_k = 0$

\subsubsection{بخش b}
\begin{center}
	\includegraphics[width = 0.5 \textwidth]{images/3.png}
\end{center}


$$\omega_0 = \frac{2\pi}{T}$$

$$a_k = \frac{1}{T_0} \int_{-T_0/2}^{T_0/2} x(t) e^{-j k \frac{2\pi}{T_0} t} dt$$

برای $k=0$ به طور جداگانه محاسبه کرده و داریم:

$$\boxed{a_0 = \frac{1}{T_0} \int_{-T_0/2}^{T_0/2} x(t) dt = 0}$$

برای باقی موارد داریم:

$$a_k = \frac{1}{T_0} \int_{-T_0/2}^{T_0/2} x(t) e^{-j k \frac{2\pi}{T_0} t} dt = \frac{1}{T_0}(\int_{-T_0/2}^{0} (-A) e^{-j k \frac{2\pi}{T_0} t} dt + \int_{0}^{T_0/2} (A) e^{-j k \frac{2\pi}{T_0} t} dt )$$

$$= \frac{1}{T_0} (\frac{A T_0 j (-1 + e^{j k \pi})}{2 k \pi} + \frac{- A  T_0 j (1 - e^{j k \pi})}{2 k \pi})$$

$$\boxed{= \frac{A j e^{-j k \pi} (-1 + e^{j k \pi})^2}{2 k \pi}}$$


\subsubsection{بخش c}

دوره تناوب پایه $|\sin (x)|$ برابر $\pi$ است و عملا مانند $\sin$ مثبتی بین $0$ تا $\pi$ است که در همه تناوب‌هایش تکرار می‌شود. در نتیجه باید براساس این تناوب حل کرد.

$$a_k = \frac{1}{\pi} \int_{0}^{\pi} |\sin (x)| e^{- 2 j k x} dx = \frac{1}{\pi}\int_{0}^{\pi} \sin (x) e^{- 2 j k x} dx $$

برای ضریب $a_0$ داریم:

$$a_0 = \frac{1}{\pi} \int_{0}^{\pi} \sin (x) dx = \frac{2}{\pi}$$


برای سایر ضرایب داریم:

$$a_k = \frac{1}{\pi}\int_{0}^{\pi} \frac{1}{2 j} (e^{i x} - e^{-ix}) e^{- 2 j k x}  dx$$

$$=\frac{1}{2 \pi} \left(\frac{e^{-j (2 k-1) x}}{2 k-1}-\frac{e^{-j (2 k+1) x}}{2 k+1}\right)|_{0}^{\pi}$$

$$=\frac{1}{2} \left(\frac{1}{2 k+1}-\frac{1}{2 k-1}\right)+\frac{1}{2} \left(\frac{e^{-j \pi  (2 k-1)}}{2 k-1}-\frac{e^{-j \pi  (2 k+1)}}{2 k+1}\right)$$

$$=\boxed{\frac{1+e^{-2 j \pi  k}}{1-4 k^2}}$$

\subsection{سوال سوم}

\subsubsection{بخش a}



$$x(t) = 
+-2j e^{-2j \omega_0 t}
+-1j e^{-1j \omega_0 t}
+1j e^{1j \omega_0 t}
+2j e^{2j \omega_0 t}
$$

$$=
-\frac{4}{2j}(e^{2j \omega_0 t} - e^{-2j \omega_0 t})
-\frac{2}{2j}(e^{j \omega_0 t} - e^{-j \omega_0 t})$$

$$\boxed{= -4 \sin(2 \omega_0 t) -2 \sin( \omega_0 t)}$$


\subsubsection{بخش b}

عبارت مورد نظر باید ما را به یاد سری فوریه قطار ضربه بیندازد.

$$\delta_{T_0} (t) = \sum_{k=-\infty}^{\infty} c_k e^{jk\omega_0 t}$$

برای ضرایب فوریه چنین چیزی داریم:
$$
c_{k}=\frac{1}{T_{0}} \int_{-T_{0} / 2}^{T_{0} / 2} \delta(t) e^{-j k \omega_{0} t} d t=\frac{1}{T_{0}}
$$

$$
\delta_{T_{0}}(t)=\sum_{k=-\infty}^{\infty} \delta\left(t-k T_{0}\right)=\frac{1}{T_{0}} \sum_{k=-\infty}^{\infty} e^{j k \omega_{0} t} \quad \omega_{0}=\frac{2 \pi}{T_{0}}
$$


با توجه به این موضوع برای چیزی که در صورت سوال داده شده، می‌توانیم آن را معادل با 

$$
z(t)=\sum_{k=-\infty}^{\infty} e^{j k \omega_{0} t} \delta(t-T_0 k+2 k)
$$

بدانیم.

در عبارت بالا $2k$ برای زوج سازی و سپس $e$ برای شیفت فرکانسی اضافه شده است که باعث بشود که تنها عبارت‌های فرد $1$ بمانند و عبارت‌های زوج $0$ شوند.



\newpage

\subsection{سوال چهارم}

در سوال نمادهای $e_k$ و $d_k$ استفاده شده است ولی برای راحتی کار و از آن جایی که کلا دو سیگنال اصلی داریم، از $a_k$ و $b_k$ در جواب استفاده شده است.

$$
x_1[n] x_2[n]=\sum_{k=0}^{N_0-1} \sum_{l=0}^{N_0-1} a_{k} b_{l} e^{j(2 \pi / N_0)(k+l) n}
$$

$$
x_1[n] x_2[n]=\sum_{k=0}^{(N_0-1)} \sum_{l^{\prime}=k}^{(k+N_0-1)} a_{k} b_{l^{\prime}-k} e^{j(2 \pi / N_0)^{\prime} n}
$$

با توجه به متناوب بودن
$b_{l' -k}$
و
$e^{j 2\pi /N_0 l' n}$
داریم:

$$
x_1[n] x_2[n]=\sum_{k=0}^{N_0-1} \sum_{l^{\prime}=0}^{N_0-1} a_{k} b_{l^{\prime}-k} e^{j(2 \pi / N_0) t^{\prime} n}=\sum_{l=0}^{N_0-1}\left[\sum_{k=0}^{N_0-1} a_{k} b_{l-k}\right] e^{j(2 \pi / N_0) l n}
$$

پس

$$
c_{k}=\sum_{t=0}^{N_0-1} a_{k} b_{l-k}
$$

و معادلا:
$$
c_{k}=\sum_{k=0}^{N_0-1} b_{k} a_{l-k}
$$

برای اثبات رابطه پارسوال داریم:


$$
N_0 \sum_{l=\langle N_0\rangle} a_{l} b_{k-l}=\sum_{\langle N_0\rangle} x_1[n] x_2[n] e^{-j(2 \pi / N_0) k n}
$$

با قرار دادن $k=0$ داریم:

$$
N_0 \sum_{l=\langle N_0\rangle} a_{l} b_{-1}=\sum_{n= \langle N_0\rangle} x_1[n] x_2[n]
$$

در نتیجه:

$$
\frac{1}{N_{0}} \sum_{n=0}^{N_{0}-1} x[n]=\sum_{k=0}^{N_{0}-1} a_{k} b_{-k}
$$

\newpage
\subsection{سوال پنجم}

در نتیجه سوال قبل قرار می‌دهیم:

$$x_2[n] = x^*_1[n]$$

در نتیجه این موضوع داریم:

$$b_k = a^*_{-k}$$

پس

$$
\frac{1}{N_{0}} \sum_{n=0}^{N_{0}-1} x[n]=\sum_{n=0}^{N_{0}-1} a_{k} n_{-k}
$$

$$
\frac{1}{N_{0}} \sum_{n=0}^{N_{0}-1} x_1[n] x_1^*[n]=\sum_{k=0}^{N_{0}-1} a_{k} a^*_{k}
$$

بنابراین:

$$
\sum_{k=\langle n_0\rangle}\left|a_{k}\right|^{2}=\frac{1}{N_0} \sum_{n=\langle N_0\rangle}|x[n]|^{2} .
$$

\newpage
\subsection{سوال ششم}
$$
f(t)=\left\{\begin{array}{lc}
	t+\frac{5}{3} & -1 \leq t<0 \\
	-t+\frac{5}{3} & 0 \leq t<2 \\
	0 & 2 \leq t<4
\end{array}\right.
$$

\subsubsection{بخش a}
$$a_0 = \frac{1}{5} \int f(t) dt = \frac{1}{5}( \int_{-1}^{0} t + 5/3 dt  + \int_{0}^{2} -t + 5/3 dt )$$
$$= \frac{1}{5}(\frac{7}{6} + \frac{4}{3}) =\boxed{ \frac{1}{2}}$$

$$a_k = \frac{1}{5} \int_{-1}^{5} f(t) e^{-j k \frac{2\pi}{5} t} dt = \frac{1}{5} (\int_{-1}^{0} (t+\frac{5}{3}) e^{-j k \frac{2\pi}{5} t} dt + \int_{0}^{2} (-t+\frac{5}{3}) e^{-j k \frac{2\pi}{5} t} dt)$$


$$\frac{1}{5} (\frac{50 i \pi  k+e^{\frac{2 i \pi  k}{5}} (-75-20 i \pi  k)+75}{12 \pi ^2 k^2} + \frac{-50 i \pi  k+e^{\frac{-4}{5} i \pi  k} (-75-10 i \pi  k)+75}{12 \pi ^2 k^2})$$$$ = \frac{e^{\frac{1}{5} (-4) i \pi  k} \left(-2 i \pi  k+30 e^{\frac{4 i \pi  k}{5}}+e^{\frac{6 i \pi  k}{5}} (-15-4 i \pi  k)-15\right)}{12 \pi ^2 k^2} $$


یا اگر روش فرمول کسینوس و سینوس را برویم داریم:


$$a_n = \frac{2}{5} \int (f(t) \cos (\frac{2\pi}{5} n t)) dt$$
$$ =\frac{2}{5} \int_{-1}^{0} (t+5/3) \cos(\frac{2 \pi}{5} n t) dt + \int_{0}^{2} (-t + 5/3) \cos (\frac{2 \pi}{5} n t) dt$$
$$= \frac{\sin ^2\left(\frac{\pi  n}{5}\right) \left(4 \pi  n \sin \left(\frac{2 \pi  n}{5}\right)+30 \cos \left(\frac{2 \pi  n}{5}\right)+45\right)}{3 \pi ^2 n^2}$$


$$b_n = \frac{2}{5} \int (f(t) \sin (\frac{2\pi}{5} n t)) dt$$
$$ =\frac{2}{5} \int_{-1}^{0} (t+5/3) \sin(\frac{2 \pi}{5} n t) dt + \int_{0}^{2} (-t + 5/3) \sin (\frac{2 \pi}{5} n t) dt$$
$$= \frac{15 \left(\sin \left(\frac{2 \pi  n}{5}\right)-\sin \left(\frac{4 \pi  n}{5}\right)\right)+4 \pi  n \cos \left(\frac{2 \pi  n}{5}\right)+2 \pi  n \cos \left(\frac{4 \pi  n}{5}\right)}{6 \pi ^2 n^2}$$



$$f(x) = a_0 + \sum_{n=1}^{N} (a_n \cos (n x) + b_n \sin (n x))$$




\subsubsection{بخش b}

کدهای مسئله به زبان پایتون در فایل 
\lr{\Verb+P1\_Q6\_b.py+}
موجود است و جواب قسمت‌های بعد براساس آن تولید شده است:

جملات مد نظر در ادامه نوشته شده اند. توجه کنید که به دلیل ویژگی‌های اعداد Floating-Point عموما ضرایبی که صفر بوده‌اند به صورت عددی ضربدر $10^{-33}$ نوشته شده‌اند.

پس از آن ابتدا در یک شکل سیگنال‌ها به ازای مقادیر $N$ به صورت جداگانه رسم شده‌اند. سپس در اشکال بعدی، به ازای هر کدام از مقادیر، نمودار آن با رنگ نارنجی روی نمودار اصلی با رنگ آبی رسم شده است.



$$a_{ 1 } = 0.772711906482877 , b_{ 1 } = 0.07175375986881644$$

$$a_{ 2 } = 0.27113541693643917 , b_{ 2 } = 0.028002784979373037$$

$$a_{ 3 } = -0.004852111911900812 , b_{ 3 } = -0.08960735823011712$$

$$a_{ 4 } = 0.004714960180721709 , b_{ 4 } = -0.010817086948420714$$

$$a_{ 5 } = 1.5195743635847465e-33 , b_{ 5 } = 0.06366197723675814$$

$$a_{ 6 } = 0.04083290139095018 , b_{ 6 } = -0.0008212742065860439$$

$$a_{ 7 } = 0.04515821107548206 , b_{ 7 } = -0.011886611106858576$$

$$a_{ 8 } = -0.01831062003374954 , b_{ 8 } = -0.023451895441009625$$

$$a_{ 9 } = -0.007676952848145469 , b_{ 9 } = -0.00338756817877342$$

$$a_{ 10 } = 1.5195743635847456e-33 , b_{ 10 } = 0.03183098861837907$$

$$a_{ 11 } = 0.017911214823682062 , b_{ 11 } = -0.0010816983697769051$$

$$a_{ 12 } = 0.023201132067024215 , b_{ 12 } = -0.008867354363816892$$

$$a_{ 13 } = -0.013610002113093692 , b_{ 13 } = -0.012990392854308745$$

$$a_{ 14 } = -0.006730131246383232 , b_{ 14 } = -0.0019169012948294058$$

$$a_{ 15 } = 1.5195743635847434e-33 , b_{ 15 } = 0.021220659078919374$$

$$a_{ 16 } = 0.01118956850490962 , b_{ 16 } = -0.000907051304878608$$

$$a_{ 17 } = 0.015464268832541013 , b_{ 17 } = -0.006821294507069879$$

$$a_{ 18 } = -0.010581176024894123 , b_{ 18 } = -0.008919232952258454$$

$$a_{ 19 } = -0.005585535368095415 , b_{ 19 } = -0.0013214190722783403$$

$$a_{ 20 } = 1.5195743635847419e-33 , b_{ 20 } = 0.015915494309189534$$

$$a_{ 21 } = 0.008076648709406326 , b_{ 21 } = -0.0007562919849315991$$

$$a_{ 22 } = 0.011564843734622392 , b_{ 22 } = -0.005507870241109013$$

$$a_{ 23 } = -0.008613443680564381 , b_{ 23 } = -0.006775588961597922$$

$$a_{ 24 } = -0.004711199325475308 , b_{ 24 } = -0.0010040831881916652$$

$$a_{ 25 } = 1.5195743635847393e-33 , b_{ 25 } = 0.012732395447351627$$

$$a_{ 26 } = 0.006300406249170883 , b_{ 26 } = -0.0006432609418058376$$

$$a_{ 27 } = 0.009225781951316991 , b_{ 27 } = -0.004609416023114295$$

$$a_{ 28 } = -0.007250921405575788 , b_{ 28 } = -0.005457578628620997$$

$$a_{ 29 } = -0.004055794714702865 , b_{ 29 } = -0.0008081706869254678$$

$$a_{ 30 } = 1.519574363584735e-33 , b_{ 30 } = 0.010610329539459687$$

$$a_{ 31 } = 0.005157489248731366 , b_{ 31 } = -0.0005579230259798711$$

$$a_{ 32 } = 0.007669732145959915 , b_{ 32 } = -0.003959686879886606$$

$$a_{ 33 } = -0.006256136875045252 , b_{ 33 } = -0.004566755490351168$$

$$a_{ 34 } = -0.003553802680498653 , b_{ 34 } = -0.0006755979057181989$$

$$a_{ 35 } = 1.5195743635847316e-33 , b_{ 35 } = 0.009094568176679736$$

$$a_{ 36 } = 0.004362360888597353 , b_{ 36 } = -0.0004918855395621645$$

$$a_{ 37 } = 0.006561005399865466 , b_{ 37 } = -0.0034690828700930944$$

$$a_{ 38 } = -0.005499407016535415 , b_{ 38 } = -0.0039249666296024815$$

$$a_{ 39 } = -0.003159413834482431 , b_{ 39 } = -0.0005800860028377107$$

$$a_{ 40 } = 1.5195743635847268e-33 , b_{ 40 } = 0.007957747154594767$$

$$a_{ 41 } = 0.003778044173238204 , b_{ 41 } = -0.00043950225495442116$$

$$a_{ 42 } = 0.005731421240004929 , b_{ 42 } = -0.003085957785323703$$

$$a_{ 43 } = -0.004905005087522983 , b_{ 43 } = -0.0034408366129256586$$

$$a_{ 44 } = -0.0028423248893797424 , b_{ 44 } = -0.0005080735752180709$$

$$a_{ 45 } = 1.5195743635847215e-33 , b_{ 45 } = 0.00707355302630646$$

$$a_{ 46 } = 0.0033308907366752464 , b_{ 46 } = -0.00039703351695859266$$

$$a_{ 47 } = 0.0050875680137263376 , b_{ 47 } = -0.0027786711713284526$$

$$a_{ 48 } = -0.004426026511190478 , b_{ 48 } = -0.003062743915561243$$

$$a_{ 49 } = -0.0025822630276678576 , b_{ 49 } = -0.0004518742494494454$$

$$a_{ 50 } = 1.5195743635847159e-33 , b_{ 50 } = 0.006366197723675813$$


\begin{center}
	\includegraphics[width = 1.0 \textwidth]{images/6-1.pdf}
\end{center}



\begin{center}
	\includegraphics[width = 1.0 \textwidth]{images/6-2.pdf}
\end{center}

\subsubsection{بخش c}

کد این بخش در فایل
\lr{\Verb+P1\_Q6\_c+}
قرار دارد.



\begin{center}
	\includegraphics[width = 1.0 \textwidth]{images/6-3.pdf}
\end{center}

\newpage
\section{تبدیل فوریه}

\subsection{سوال اول}

$$X(jw) = \int_{-\infty}^{\infty} x(t) e^{-j \omega t} dt$$

\subsubsection{بخش a}

$$e^{- a |t|} \sin \omega_0 t$$

$$X(jw) = \int_{-\infty}^{\infty} e^{- a |t|} \sin (\omega_0 t) e^{-j \omega t}  dt$$
$$\int_0^{\infty}  \sin (\omega_0 t) e^{(-jw -a) t} + \int_{-\infty}^{0}  \sin (\omega_0 t) e^{(-jw + a)t}$$

$$= \int_{0}^{\infty} e^{-at} \sin(\omega_0 t) (e^{-jwt} - e^{jwt})$$

$$= -2j \int_{0}^{\infty} e^{at} \sin(\omega_0 t) \sin(\omega t)$$


$$= j \int_{0}^{\infty} e^{at} (\cos ((\omega_0 + \omega) t) - \cos((\omega_0 - \omega )t))$$

$$= j(e^{a t} \left(\frac{a \cos (t (\omega_0-\omega))+(\omega_0-\omega) \sin (t (\omega_0-\omega))}{a^2+(\omega_0-\omega)^2}-\frac{a \cos (t (\omega_0+\omega))+(\omega_0+\omega) \sin (t (\omega_0+\omega))}{a^2+(\omega_0+\omega)^2}\right))|_{0}^{\infty}$$

با شرط $a<0$ داریم:

$$= \frac{4 a \omega_0 \omega j}{\left(a^2+\omega_0^2\right)^2+2 \omega^2 (a-\omega_0) (a+\omega_0)+\omega^4}$$

\newpage


\subsubsection{بخش b}

$$
X(j \omega)=\int_{-1}^{1}(1+\cos (\pi t)) e^{-j \omega t} d t
$$

$$
\begin{gathered}
	X(j \omega)=\int_{-1}^{1} e^{-j \omega t} d t+\int_{-1}^{1} \frac{e^{j \pi t}+e^{-j \pi t}}{2} e^{-j \omega t} d t \\
	X(j \omega)=\left.\frac{e^{-j \omega t}}{-j \omega}\right|_{-1} ^{1}+\left.\frac{1}{2}\left(\frac{e^{j(\pi-\omega)}}{j(\pi-\omega)}+\frac{e^{-j(\pi+\omega) t}}{-j(\pi+\omega)}\right)\right|_{-1} ^{1} \\
	X(j \omega)=\frac{e^{-j \omega}-e^{j \omega}}{-j \omega}+\frac{1}{2}\left(\frac{e^{j(\pi-\omega)}-e^{j(\pi-\omega)}}{j(\pi-\omega)}+\frac{e^{-j(\pi+\omega)}-e^{j(\pi+\omega)}}{-j(\pi+\omega)}\right) \\
	X(j \omega)=\frac{2}{\omega} \cdot \frac{e^{j \omega}-e^{-j \omega}}{2 j}+\frac{1}{\pi-\omega} \cdot \frac{e^{j(\pi-\omega)}-e^{j(\pi-\omega)}}{2 j}+\frac{1}{\pi+\omega} \cdot \frac{e^{j(\pi+\omega)}-e^{-j(\pi+\omega)}}{2 j} \\
	X(j \omega)=\frac{2 \sin \omega}{\omega}+\frac{\sin (\pi-\omega)}{\pi-\omega}+\frac{\sin (\pi+\omega)}{\pi+\omega}
\end{gathered}
$$

$$\boxed{X(j\omega) = \frac{2 \sin \omega}{\omega}+\frac{\sin (\pi-\omega)}{\pi-\omega}+\frac{\sin (\pi+\omega)}{\pi+\omega}}$$

\subsubsection{بخش c}


می‌دانیم تبدیل فوریه $e^{a|t|}$ به صورت
$\frac{2a}{a^2 + \omega^2}$
است.
اثبات:

$$
\begin{gathered}
	x(t)=e^{-d t \mid}=\left\{\begin{array}{ll}
		e^{-a t} & t>0 \\
		e^{a t} & t<0
	\end{array}\right. \\
	X(\omega)=\int_{-\infty}^{0} e^{a t} e^{-j \omega t} d t+\int_{0}^{\infty} e^{-a t} e^{-j \omega t} d t \\
	=\int_{-\infty}^{0} e^{(a-j \omega) t} d t+\int_{0}^{\infty} e^{-(a+j \omega) t} d t \\
	=\frac{1}{a-j \omega}+\frac{1}{a+j \omega}=\frac{2 a}{a^{2}+\omega^{2}}
\end{gathered}
$$

همچنین می دانیم که تبدیل فوریه
$t f(t)$
تبدیل فوریه برابر است با:
$j \frac{d}{d\omega} F(\omega)$

پس در این جا هم جواب

$$j \frac{d}{d \omega } \frac{2a}{a^2 + \omega^2} = -\frac{4 a  j \omega}{\left(a^2+\omega^2\right)^2}$$


\subsubsection{بخش d}

$$X(jw) = \int_{-\infty}^{\infty} \cos (\omega_0 t) u(t) e^{-j \omega t} dt$$

$$X(jw) = \int_{0}^{\infty} \cos (\omega_0 t) e^{-j \omega t} dt$$

$$X(jw) = \frac{1}{2} \int_{0}^{\infty} (e^{j \omega_0 t} + e^{-j \omega_0 t}) e^{-j \omega t} dt$$
$$= -\frac{j \omega}{\omega^2-{\omega_0}^2}$$


\subsubsection{بخش e}

$$
\Delta(t)=\left\{\begin{array}{lr}
	1-2|t| & 0 \leq t \leq 1 / 2 \\
	0 & \text { otherwise }
\end{array}\right. = 
\left\{\begin{array}{lr}
	1-2t & 0 \leq t \leq 1 / 2 \\
	0 & \text { otherwise }
\end{array}\right.
$$


$$
\begin{aligned}
	F(\omega) &=\int_{-\infty}^{\infty} \Delta(t) e^{-j \omega t} d t \\
	&=\int_{0}^{1 / 2}(1-2 t) e^{-j \omega t} d t \\
	&=\left.\frac{j \omega(2 t-1)+2}{(j \omega)^{2}} e^{-j \omega t}\right|_{t=0} ^{1 / 2} \\
	&=\frac{2-j \omega-2 e^{-j \omega / 2}}{\omega^{2}}
\end{aligned}
$$

\subsubsection{بخش f}

$$
x(t)=\left\{\begin{aligned}
	1 & \text { if } 1 \leq|t| \leq 3 \\
	-1 & \text { if }|t|<1 \\
	0 & \text { otherwise }
\end{aligned}\right.
$$

می‌دانیم که تبدیل فوریه سیگنال مستطیلی بین $-1/2$ تا $1/2$ به صورت:
$\frac{\sin \frac{\omega }{2}}{\frac{\omega}{2}} = sinc (\omega/2)$
است. اگر سیگنال مستطیلی ذکر شده را با نماد $\Pi(t)$ نمایش بدهیم، عبارت بالا
$\Pi(t/6) - 2\Pi (t/2)$
است. در نتیجه
$$F(\omega) = 6sinc(6 \omega/2) - 4 sinc(2 \omega /2) = 6sinc(3\omega) - 4 sinc(\omega)$$

\newpage

\subsection{سوال دوم}

\subsubsection{بخش a}

$$
F(\omega)=\frac{16-16 j \omega+4 \omega^{2}-4 j \omega^{3}}{54+81 j \omega+18 \omega^{2}+31 j \omega^{3}-6 \omega^{4}}
$$

$$F(\omega) = \frac{4 (-2 + j\omega) (-1 + j\omega) (2 + j\omega)}{-(3 + j\omega)^2 (-3 + 2 j\omega) (2 + 3 j\omega)}$$

$$= \frac{80}{63 (j\omega+3)^2}+\frac{28}{1053 (2 j\omega-3)}+\frac{640}{637 (3 j\omega+2)}-\frac{4028}{3969 (j\omega+3)}$$

$$\frac{80}{63} t e^{-3t} u(t) +\frac{-14}{1053}e^{\frac{3}{2}t}u(-t)+\frac{640}{1911}e^{\frac{-2}{3} t}u(t) + \frac{4028}{3969} e^{-3t}u(t) $$

\subsubsection{بخش b}

$$F(j\omega) = 2\pi j \omega e^{-|\omega|}$$


$$x(t) =\frac{1}{2\pi}\int_{-\infty}^{\infty} 2\pi j \omega e^{-|\omega|} e^{j \omega t} d \omega$$

$$ \int_{-\infty}^{0} j\omega e^{\omega} e^{j \omega t} d\omega + \int_{0}^{\infty} j\omega e^{-\omega} e^{j\omega t} d\omega$$

$$=\frac{i}{(t-i)^2} +(-\frac{i}{(t+i)^2}) $$

$$=-\frac{4 t}{\left(t^2+1\right)^2}$$

\newpage
\subsection{سوال سوم}
\newpage

\subsection{سوال چهارم}

$$h(t)=\frac{\sin (10 \pi  t)-\sin (6 \pi  t)}{2 \pi  t}$$

$$H(j\omega) = 
\left\{\begin{array}{ll}
	\frac{1}{2}, & |\omega|<10\pi \\
	0, & |\omega|>10\pi
\end{array}\right.
 -
 \left\{\begin{array}{ll}
 	\frac{1}{2}, & |\omega|<6\pi \\
 	0, & |\omega|>6\pi
 \end{array}\right.
 $$
 
 
 $$x(t) = \frac{1}{2} (\cos (5 \pi  t)+\cos (9 \pi  t))$$
 
 
 $$X(j\omega) = \frac{\pi}{2} (\delta(\omega-5\pi) + \delta(\omega + 5\pi)) + \frac{\pi}{2} (\delta(\omega-9\pi) + \delta(\omega + 9\pi))$$
 
 با ضرب $H$ در $X$ ، عبارت داری $5\pi$ که فقط در $5\pi$ مقدار دارد، در هر دو حالت $H$ شامل حالت $\frac{1}{2}$ شده و صفر می‌شود. ولی عبارت دومی فقط در حالت $|w|<10\pi$ صدق می‌کند و نصف می‌شود. در نتیجه:
 
 $$Y(j\omega) =\frac{\pi}{4} (\delta(\omega-9\pi) + \delta(\omega + 9\pi))$$
 
 پس
 
 $$y(t) = \frac{1}{4} \cos (9 \pi t)$$
\end{document}



